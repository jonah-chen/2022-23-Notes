\documentclass[a4paper]{article}
\usepackage{stdtemplate}

\title{MIE407 Nuclear Reactor Theory \& Design}
\author{Jonah Chen}
\date{}
\begin{document}
\maketitle
\sffamily
\textbf{Nuclear Stability:}
\begin{itemize}
    \item Stability of nucleus is the result of the balance between the strong nuclear force and the electromagnetic force.
    \item Range of the strong nuclear force is about \SI{1}{fm} (femtometre, \SI{e-13}{cm})
    \item The strong nuclear force acts equally between protons and neutrons, but neutrons also reduce repulsions between protons by pushing them further apart from each other.
    \item Fermi approx: $r = R_0 A^{1/3}$ where $R_0 = \SI{1.2}{fm}$
    \item The strong nuclear force is repulsive at very small distances, which contributes to the incompressibility of the nucleus.
    \item The strong nuclear force is insignificant at larger than around 4 proton diameters.
\end{itemize}
\textbf{Nuclear Binding Energy:}
\begin{itemize}
    \item The binding energy of a nucleus can be found by calculating the mass defect of the nucleus compared to the mass of the unbound nucleons.
    \begin{equation}
        E_B=[Zm_p + (A-Z)m_n - m] c^2\equiv \Delta M\times c^2
    \end{equation}
    \item The binding energy is usually expressed as the average binding energy per nucleon $\varepsilon = E_B/A$.
    \item \SI{1}{u}$=$\SI{931.5}{MeV/c^2}
\end{itemize}
\textbf{Compound nucleus decay modes:}
\begin{itemize}
    \item Neutron capture $(n,\gamma)$: Nucleus decays to a lower energy state by emitting some gamma rays. $n+ ^AX\to ^{A+1}X^*\to ^{A+1}X+\gamma$
    \item Elastic scattering $(n,n')$: Neutron is re-emitted after leaving the nucleus in the ground state.
    \item Inelastic scattering $(n,n'\gamma)$: Neutron (usually high energy) is re-emitted at a lower energy, leaving the nucleus in an excited state. Then, the nucleus emits some gamma rays to return to the ground state.
    \item Fission $(n,f)$: Nucleus splits into two fragments with approx. $2:3$ mass ratio.
    \item Particle emission $(n,\alpha),(n,p),(n,kn)$: A particle other than a neutron, or multiple neutrons are emitted. Only occurs with very high energy neutrons. Hence, only a small number of $(n,2n)$ reactions occur in nuclear reactors.
\end{itemize}
\textbf{Nuclear Fission}
\begin{itemize}
    \item When a thermal neutron ($\leq$\SI{1}{eV}) is absorbed by a U-235 nucleus, it forms a compound nucleus that fissions with $\approx84\%$ probability.
    \item Fission is a threshold reaction. In cases for U-235, the threshold energy is lower than the energy gained by binding the extra neutron. Hence, fissions may occur by neutrons with any energy. However, for U-238, the threshold energy is about 1 MeV greater than the energy gained by binding the extra neutron. Hence, fissions only occur by neutrons with kinetic energy greater than 1 MeV, which rarely occurs in even a fast reactor.
    \item Under thermal conditions (\SI{0.0253}{eV} or \SI{2200}{m/s}) the distribution of fission products is a strong bimodal distribution (Peaks at $A=96,135$).
    \item Most neutrons (prompt neutrons) emitted in fission are released at the instant of fission.
    \item A small number of fission fragments also emit neutrons (delayed neutrons). These compose of $<1\%$ of the total neutrons, but they are significant in the transient behavior of the reactor. 
    \item Delayed neutrons result from high nuclear excitation of the daughter (when the excess energy exceed the nuclear binding energy of the neutron)
    \item Between 0 to 7 prompt neutrons may be emitted by fission, on average around 2.4. This is very important.
    \item The fission neutron spectrum is approximately $\chi(E)=0.484\sinh(\sqrt{2E})e^{-E}\si{MeV}^{-1}$. Integrate this over a interval of $E$ to obtain the probability of energy in that interval. (Average \SI{2}{MeV})
    \item Delayed neutrons are much lower energy (\SI{400}{keV})
\end{itemize}

\end{document}