\documentclass[a4paper]{article}
\usepackage{stdtemplate}

\title{MIE407 Nuclear Reactor Theory \& Design}
\author{Jonah Chen}
\date{}
\begin{document}
\maketitle 
\sffamily
\textbf{Nuclear Stability:}
\begin{itemize}
    \item Stability of nucleus is the result of the balance between the strong nuclear force and the electromagnetic force.
    \item Range of the strong nuclear force is about \SI{1}{fm} (femtometre, \SI{e-13}{cm})
    \item The strong nuclear force acts equally between protons and neutrons, but neutrons also reduce repulsions between protons by pushing them further apart from each other.
    \item Fermi approx: $r = R_0 A^{1/3}$ where $R_0 = \SI{1.2}{fm}$
    \item The strong nuclear force is repulsive at very small distances, which contributes to the incompressibility of the nucleus.
    \item The strong nuclear force is insignificant at larger than around 4 proton diameters.
\end{itemize}
\textbf{Nuclear Binding Energy:}
\begin{itemize}
    \item The binding energy of a nucleus can be found by calculating the mass defect of the nucleus compared to the mass of the unbound nucleons.
    \begin{equation}
        E_B=[Zm_p + (A-Z)m_n - m] c^2\equiv \Delta M\times c^2
    \end{equation}
    \item The binding energy is usually expressed as the average binding energy per nucleon $\varepsilon = E_B/A$.
    \item \SI{1}{u}$=$\SI{931.5}{MeV/c^2}
\end{itemize}
\textbf{Compound nucleus decay modes:}
\begin{itemize}
    \item Neutron capture $(n,\gamma)$: Nucleus decays to a lower energy state by emitting some gamma rays. $n+ ^AX\to ^{A+1}X^*\to ^{A+1}X+\gamma$
    \item Elastic scattering $(n,n')$: Neutron is re-emitted after leaving the nucleus in the ground state.
    \item Inelastic scattering $(n,n'\gamma)$: Neutron (usually high energy) is re-emitted at a lower energy, leaving the nucleus in an excited state. Then, the nucleus emits some gamma rays to return to the ground state.
    \item Fission $(n,f)$: Nucleus splits into two fragments with approx. $2:3$ mass ratio.
    \item Particle emission $(n,\alpha),(n,p),(n,kn)$: A particle other than a neutron, or multiple neutrons are emitted. Only occurs with very high energy neutrons. Hence, only a small number of $(n,2n)$ reactions occur in nuclear reactors.
\end{itemize}
\textbf{Nuclear Fission}
\begin{itemize}
    \item When a thermal neutron ($\leq$\SI{1}{eV}) is absorbed by a U-235 nucleus, it forms a compound nucleus that fissions with $\approx84\%$ probability.
    \item Fission is a threshold reaction. In cases for U-235, the threshold energy is lower than the energy gained by binding the extra neutron. Hence, fissions may occur by neutrons with any energy. However, for U-238, the threshold energy is about 1 MeV greater than the energy gained by binding the extra neutron. Hence, fissions only occur by neutrons with kinetic energy greater than 1 MeV, which rarely occurs in even a fast reactor.
    \item Under thermal conditions (\SI{0.0253}{eV} or \SI{2200}{m/s}) the distribution of fission products is a strong bimodal distribution (Peaks at $A=96,135$).
    \item Most neutrons (prompt neutrons) emitted in fission are released at the instant of fission.
    \item A small number of fission fragments also emit neutrons (delayed neutrons). These compose of $<1\%$ of the total neutrons, but they are significant in the transient behavior of the reactor. 
    \item Delayed neutrons result from high nuclear excitation of the daughter (when the excess energy exceed the nuclear binding energy of the neutron)
    \item Between 0 to 7 prompt neutrons may be emitted by fission, on average around 2.4. This is very important.
    \item The fission neutron spectrum is approximately $\chi(E)=0.484\sinh(\sqrt{2E})e^{-E}\si{MeV}^{-1}$. Integrate this over a interval of $E$ to obtain the probability of energy in that interval. (Average \SI{2}{MeV})
    \item Delayed neutrons are much lower energy (\SI{400}{keV})
    \item Reaction Rate stuff
    \begin{itemize}
        \item Intensity $I=nv$ (\si{1/cm^2s})
        \item Flux $\phi$ (\si{1/cm^2s})
        \item Number density $N=\frac{\rho N_A}{M}$ (\si{1/cm^3})
        \item Microscopic cross section $\sigma$ (\si{1/cm^2})
        \item Macroscopic cross section $\Sigma=N\sigma$ (\si{1/cm})
        \item Reaction rate $R=\Sigma\phi$ (\si{1/cm^3s})
    \end{itemize}
    \item \textit{Passive} or non-multiplying media are characterized by the scattering ($\sigma_s$) and capture ($\sigma_c$) cross sections.
    \item \textit{Multiplying} media contain at least one fissile or fissionable (fissile at high energy only) isotope and are further characterized by fission cross-section $\sigma_f$ and $\nu$.
\end{itemize}
\section{Propogation of Neutrons in a Passive Medium}
\begin{itemize}
    \item Due to interactions with the medium, the initial intensity $I_0$ decreases to $I(x)$ at depth $x$.
    \item The rate of decrease is the reaction rate $I'=-R=\Sigma I$, with solution $I(x)=I_0e^{-\Sigma x}$. $\Sigma$ is the total cross-section. 
    \item Understood in probability terms, $e^{-\Sigma x}$ is the probability that a neutron survive to depth $x$. Also, for very small $\Delta x$, $\Sigma\Delta x$ is the probability that a neutron will interact in $\Delta x$.
    \item Finally, $\Sigma e^{-\Sigma x}\Delta x$ is the probability that a neutron will interact between distance $x$ and $x+\Delta x$.
    \item The \bluebf{mean free path} is defined as the average distance a neutron travels before interacting \begin{equation}
        \lambda=\int_0^\infty xp(x)\dd x=\int_0^\infty x\Sigma e^{-\Sigma x}\dd x=\frac{1}{\Sigma}
    \end{equation}
    Note that the mean free path can be used for a single reaction type also, e.g. $\lambda_s, \lambda_c, \lambda_f$.
    \item Point neutron source. The intensity of the source is $\dot S$. 
    \item In empty space, $\phi(r)=\frac{\dot S}{4\pi r^2}$.
    \item In medium, $\phi(r)=\frac{\dot S}{4\pi r^2}e^{-\Sigma r}$
    \item If there is scattering, these equations will underestimate the flux.
\end{itemize}
\section{Neutron Flux and Current}
\begin{itemize}
    \item In analyzing a nuclear reactor, we are interested in the neutron population at any position/time, and the rates of all nuclear reaction at any position/time.
    \item The neutron density is $n(\mathbf r, \mathbf v, t)$. It is useful to express the velocity vector as the speed $v$ and the unit direction $\mathbf\Omega$. We also use kinetic energy (which contains the same information as the speed $E=\frac{1}{2}mv^2$) i,e, $n(\mathbf r,E,\mathbf\Omega, t)$.
    \item $n(\mathbf r,E,\mathbf\Omega, t)$ is a density function with units \si{1/cm^3\electronvolt\steradian}. $n(\mathbf r,E,\mathbf\Omega, t)\dd V\dd E\dd\Omega$.
    \item With $\mu:=\cos\theta,$ we have $\dd\Omega=-\dd\varphi\dd\mu/4\pi,$ when we integrate over the all solid angles, we can integrate $\varphi$ from $0$ to $2\pi$ and $\mu$ from $1$ to $-1$.
    \item Here, we define $\dd\Omega=\dd S/4\pi=-\dd\varphi\dd(\cos\theta)/4\pi$. Note that $\int_{4\pi}\dd\Omega=\frac{1}{4\pi}\int_0^{2\pi}\dd\varphi\int_{-1}^1\dd\mu=1$
    \item The \bluebf{angular neutron density} $n(\mathbf r, E, \mathbf\Omega)$ is an important reactor parameter used in the definition of neutron flux and current.
    \item The integral of the angular neutron density over all solid angles is the \bluebf{neutron density} in \si{cm^{-3}\electronvolt^{-1}}\begin{equation}
        n(\mathbf r, E)=\int_{4\pi}n(\mathbf r, E, \mathbf\Omega)\dd\Omega.
    \end{equation}
    \item If some quantity $q$ with density $\rho$ can flow by a velocity field $\mathbf v,$ we can define the \bluebf{vector flux}\begin{equation}
        \mathbf j=\rho\mathbf v.
    \end{equation} We can define the \bluebf{neutron flux density} as\begin{equation}
        \mathbf\phi(\mathbf r, E,\mathbf\Omega)=n(\mathbf r, E,\mathbf\Omega)\mathbf v(\mathbf r, E,\mathbf\Omega).
    \end{equation}
    \item The magnitude of vector flux density is the combined distance traveled by these neutrons (\textbf{this is important for determining reaction rate}) is defined as the \bluebf{scalar neutron flux},\begin{equation}
        \Phi(\mathbf r,E)=\int_{4\pi}\mathbf\phi(\mathbf r, E,\mathbf\Omega)\dd\Omega=v\int_{4\pi}n(\mathbf r, E,\mathbf\Omega)\dd\Omega.
    \end{equation}
    \item The \bluebf{neutron (net) current density} (\textbf{this is important for determining neutron transport}) is\begin{equation}
        \mathbf J(\mathbf r,E)=\int_{4\pi}\mathbf vn(\mathbf r,E,\mathbf\Omega)\dd\Omega=v\int_{4\pi}\hat{\mathbf\Omega}\, n(\mathbf r,E,\mathbf\Omega)\dd\Omega\label{eq:neutron_current_density}
    \end{equation}
    \item Since the integral in \eqref{eq:neutron_current_density} is performed over all direction, which yields a vector that points in a particular direction. The direction of $\mathbf J$ represents the direction of \redbf{net} neutron flow.
    \item We can specify a unit vector $\hat n$ that is normal to $\dd A,$ then $\mathbf J\cdot\hat n\dd A$ is the net number of neutrons crossing the surface $\dd A$ per unit energy and unit time.
\end{itemize}

\section{Neutron Diffusion Equation}
\begin{itemize}
    \item We will first try to understand how neutrons flows through an volume element $\Delta V=\Delta x\Delta y\Delta z$ within a passive medium. In a passive medium, there are two types of interactions (absorption and scattering).
    \item For sake of simplicity, we will assume a partial current $\mathbf J=J_x\hat x.$ Due to interactions, we have \begin{equation}
        \frac{J_x(x+\Delta x,y,z)-J_x(x,y,z)}{\Delta x}\to \partialderivative{J_x}{x}\dd V
    \end{equation}
    We can repeat this for the $J_y$ and $J_z$ to obtain the \bluebf{leakage} form
    \begin{equation}
        \dd\mathcal{L}=\mathrm{div\,}\mathbf J\dd V. \label{eq:leakage}
    \end{equation}
    \item At steady state, the net leakage \eqref{eq:leakage} represents a mismatch between neutron gain and loss.
    \item The overall neutron balance in $\dd V$ becomes \begin{equation}
        \dd\mathcal{L}=(Q-R_a)\dd V
    \end{equation}
    where $Q$ is the source density in \si{cm^{-3}s^{-1}} and $R_a=\Sigma_a\phi$ is the rate of absorption per unit volume.
    \begin{equation}
        \mathrm{div\,}\mathbf J+\Sigma_a\Phi=Q.\label{eq:neutron_balance}
    \end{equation}
    equation \eqref{eq:neutron_balance} is the \bluebf{neutron balance equation}.
    \item We need an equation to relate $\mathbf J$ and $\Phi$. The most common approximation is the \bluebf{diffusion approximation}, which assumes the neutron flux (\redbf{not neutron density}) behaves like a fluid \begin{align}
        \mathbf J(\mathbf r)=-\mathbf D(\mathbf r)\nabla\Phi(\mathbf r)\label{eq:fick}
    \end{align}
    \item Here, the diffusion coefficient $\mathbf D$ has the dimensions of length. This depends on spatial position because the material in the reactor is not homogeneous. The diffusion coefficient is related to the cross-sections of the medium \begin{equation}
        D=\frac{\lambda_{tr}}{3}=\frac{1}{\Sigma_{tr}},
    \end{equation}
    where $\lambda_{tr}$ is the transport mean free path in \si{cm}.
\end{itemize}
\subsection{Transport Cross-Section}
\begin{itemize}
    \item The macroscopic transport cross-section is determined by the density of the medium, \begin{equation}
        \Sigma_{tr}=N\sigma_{tr}
    \end{equation}
    \item The microscopic transport cross-section accounts for both absorption and inelastic scattering \begin{equation}
        \sigma_{tr}=\sigma_a+(1-\bar\mu)\sigma_s,
    \end{equation}
    where $\bar\mu\approx\frac{2}{3A}$ is the average cosine of the scattering angle $\theta.$
    \item In the center of mass frame, the scattering is isotropic, but in the lab frame it is not due to conservation of momentum. $0\leq\theta<\SI{180}{\degree}$ or $0\leq\bar\theta<\SI{90}{\degree}.$ As $\mu=\cos\theta,1\geq\mu>0.$
    \item Typically, $\bar\mu<<1$ for heavier atoms $A>>1,$ so $\sigma_{tr}=\sigma_a+\sigma_s=\sigma_t.$ In this course, \textbf{make this approximation unless otherwise told}.
\end{itemize}
\subsection{Limitations of the Diffusion Approximation}
\begin{itemize}
    \item Underlying diffusion theory is an approximation that expresses the scattering kernel as an expression in Legendre polynomials, in which the first order term are used
    \begin{align}
        \Sigma_s(z,\mathbf\Omega\to\mathbf\Omega')=\Sigma_s(z,\Omega\cdot\Omega')=\frac{1}{2\pi}\sum_{l=0}^\infty\frac{2l+1}{2}P_l(\mu)\Sigma_s(z)
    \end{align}
    where $\mu=\cos(\mathbf\Omega\cdot\mathbf\Omega')$ and $P_l$ is the Legendre polynomial of order $l.$
    \item This approximation is good \redbf{except}:\begin{itemize}
        \item Near a strong source or absorber
        \item At the boundary between two media with dissimilar neutron diffusion characteristics
    \end{itemize}
\end{itemize}

\end{document}