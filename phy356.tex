\documentclass[12pt]{article}
\usepackage{stdtemplate}
\title{PHY356: Quantum Mechanics}
\author{Jonah Chen}
\date{Fall 2022}
\begin{document}
\maketitle
\sffamily
\section{Introduction}
Schrodinger's Equation:
\begin{align}
    i\hbar\frac{\partial}{\partial t}\psi(x, t) = \left(\frac{p^2}{2m}+V(x,t)\right)\psi(x, t)
\end{align}

Conservation of Probability:
\begin{align}
    \partialderivative{(\psi^*\psi)}{t} = \partialderivative{\psi^*}{t}\psi+\psi^*\partialderivative{\psi}{t} = -\mathrm{div} J
\end{align}

We can use the Schrodinger's equation and its complex conjugate to get the following:
\begin{align}
    \partialderivative{\psi^*}{t}\psi&=\\
    \psi^*\partialderivative{\psi}{t}&=
\end{align}

Probability Current is then 
\begin{align}
    J = -\frac{i\hbar}{2m}\left(\psi^*\nabla\psi-\psi\nabla\psi^*\right)
\end{align}
This requires the wave function $\psi$ to be continuously differentiable (for non-singular potentials), unlike the electric field which can be discontinuous on surfaces.

Question: How does the wave mechanics recapture the notion of a classical particle?

Superposition Principle: If $\psi_1$ and $\psi_2$ are two solutions to the Schrodinger equation, then $\alpha\psi_1+\beta\psi_2$ is also a solution for $\alpha, \beta\in\C$.

Simple Example:
\begin{align}
    \cos k_1x+\cos k_2x = 2\cos\left(\frac{k_1+k_2}{2}x\right)\cos\left(\frac{k_1-k_2}{2}x\right)
\end{align}
Let $k_1 = k_0+\Delta k$ and $k_2 = k_0-\Delta k$, assume $\Delta k\ll k_0$.

Fourier Decomposition:
General solution of the free-particle schrodinger equation
\begin{align}
    \psi(\mathbf r, t) = (2\pi)^{-3/2}\int \dd^3\mathbf k\, g(\mathbf k)\exp\left(i(\mathbf k\cdot\mathbf r-\omega t)\right)\psi_k
\end{align}

For simplicity, we will use $t=0$ and 1 dimension, with a normal distribution of frequencies $g(k)=\exp{-\frac{(k-k_0)^2}{2(\Delta k)^2}}$. We can perform the fourier transform to get $\psi(x)=\exp{i\Delta k^2 x^2}$

\end{document}