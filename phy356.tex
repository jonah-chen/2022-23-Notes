\documentclass[12pt]{article}
\usepackage{stdtemplate}
\title{PHY356: Quantum Mechanics}
\author{Jonah Chen}
\date{Fall 2022}
\begin{document}
\maketitle
\sffamily
\section{Introduction}
Schrodinger's Equation:
\begin{align}
    i\hbar\frac{\partial}{\partial t}\psi(x, t) = \left(\frac{p^2}{2m}+V(x,t)\right)\psi(x, t)
\end{align}

Conservation of Probability:
\begin{align}
    \partialderivative{(\psi^*\psi)}{t} = \partialderivative{\psi^*}{t}\psi+\psi^*\partialderivative{\psi}{t} = -\mathrm{div} J
\end{align}

We can use the Schrodinger's equation and its complex conjugate to get the following:
\begin{align}
    \partialderivative{\psi^*}{t}\psi&=\\
    \psi^*\partialderivative{\psi}{t}&=
\end{align}

Probability Current is then 
\begin{align}
    J = -\frac{i\hbar}{2m}\left(\psi^*\nabla\psi-\psi\nabla\psi^*\right)
\end{align}
This requires the wave function $\psi$ to be continuously differentiable (for non-singular potentials), unlike the electric field which can be discontinuous on surfaces.

Question: How does the wave mechanics recapture the notion of a classical particle?

Superposition Principle: If $\psi_1$ and $\psi_2$ are two solutions to the Schrodinger equation, then $\alpha\psi_1+\beta\psi_2$ is also a solution for $\alpha, \beta\in\C$.

Simple Example:
\begin{align}
    \cos k_1x+\cos k_2x = 2\cos\left(\frac{k_1+k_2}{2}x\right)\cos\left(\frac{k_1-k_2}{2}x\right)
\end{align}
Let $k_1 = k_0+\Delta k$ and $k_2 = k_0-\Delta k$, assume $\Delta k\ll k_0$.

Fourier Decomposition:
General solution of the free-particle schrodinger equation
\begin{align}
    \psi(\mathbf r, t) = (2\pi)^{-3/2}\int \dd^3\mathbf k\, g(\mathbf k)\exp\left(i(\mathbf k\cdot\mathbf r-\omega t)\right)\psi_k
\end{align}

For simplicity, we will use $t=0$ and 1 dimension, with a normal distribution of frequencies $g(k)=\exp{-\frac{(k-k_0)^2}{2(\Delta k)^2}}$. We can perform the fourier transform to get $\psi(x)=\exp{i\Delta k^2 x^2}$

Stationary States: 
These are eigenvectors of the hamiltonian, $H=-\frac{\hbar^2}{2m}\nabla^2+V(x)$.

We will analyze the finite square well
\begin{equation}
    V(x) = \begin{cases}
        -V_0 & \text{if } |x|<a/2\\
        0 & \text{otherwise}
    \end{cases}
\end{equation}
For bound states, plane waves will solve the schrodinger equation within the well $\phi_k(x)=A_1e^{ikx}+A_2e^{-ikx}$. Outside the well, exponentials will solve the schrodinger equation $\phi_k(x)=B_1e^{qx}+B_2e^{-qx}$. 

Note how there is a nonzero probability to find the particle outside the well, which is classically disallowed. This is one case of tunneling.

Also note the symmetry of the potential, then the bound states must be either odd or even. Then, we only need to solve for one side of the well. We will focus on the even states first.

For even states, $\psi'(0)=0$ so $ikA_1-ikA_2=0\implies A_1=A_2\equiv A$. Also note the boundary condition $\psi(a/2), \psi'(a/2)$ must be continuous. 
\begin{align}
    A\cos\frac{ka}{2}&=Be^{-qa/2}\\
    Ak\sin\frac{ka}{2}&=Bqe^{-qa/2}
\end{align}

We can divide the equations by each other and obtain
\begin{equation}
    \cot\frac{ka}{2}=\frac{k}{q}
\end{equation}

We know $k=\sqrt{2m(E+V_0)}/\hbar>0$ and $q=\sqrt{-2mE}/\hbar>0$. We define $k^2+q^2=2mV_0/\hbar^2\equiv k_0^2$. Then, $k/k_0=|cos(ka/2)$.

We can non dimensionalize the equation by letting $x=ay$. Then, $a=\sqrt{2mV_0/\hbar^2}$. We can then solve for $y$:
\begin{equation}
    \left(\frac{-\hbar^2}{2ma^2}\frac{d^2}{dy^2}-V_0\right) \psi = E\psi
\end{equation}

Recapture the classical limit for step, smooth out the potential step
\end{document}