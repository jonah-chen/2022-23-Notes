\documentclass[a4paper]{article}
\usepackage{stdtemplate}

\title{MAT351 PDE}
\author{Jonah Chen}
\date{}
\begin{document}
\maketitle
\sffamily
Transport Equation (simplest form)
\begin{align}
\frac{\partial u}{\partial t} + c\frac{\partial u}{\partial x} = 0
\end{align}
where $u:(x,t)\in\R\times\R^+\to\R, c\in\R$

Diffusion Equation (simplest form)
\begin{align}
    \frac{\partial u}{\partial t} - k \frac{\partial^2 u}{\partial x^2} = 0
\end{align}
where $k>0$

Wave Equation
\begin{align}
    \frac{\partial^2 u}{\partial t^2} - c^2\frac{\partial^2 u}{\partial x^2} = 0
\end{align}
where $c$ is the speed of the wave.


In the 19th century, the research focused on finding explicit solutions. Most notably:
\begin{enumerate}
    \item Method of characteristics (Hamilton)
    \item Fourier analysis
    \item Green's functions
\end{enumerate}
In the 20th century, numerical methods were developed for equations without explicit solutions. This is used for modelling, weather prediction, finance, etc. 

Theoretical questions are posed like uniqueness, behavior, stability, and other qualitative properties of PDEs.

\begin{definition}[Well-posedness]
    A PDE (problem) is well posed if
    \begin{enumerate}
        \item A solution exists.
        \item The solution is unique.
        \item The solution is stable. (i.e. small perturbations of initial conditions or boundary conditions leads to small perturbations of the solution)
    \end{enumerate}
\end{definition}

Classification:
\begin{enumerate}
    \item Order of PDE (highest order of derivative)
    \item Linearity/non-linearity
    \begin{equation}
        L(u) = f(x)
    \end{equation}
    where $L$ is a linear operator and $f$ is a function.
    \begin{equation}
        \partialderivative{u}{t} - k\partialderivative^2{u}{x^2} + u^3 = x^2 + 1
    \end{equation}
\end{enumerate}

\end{document}
