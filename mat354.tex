\documentclass[a4paper,12pt]{article}
\usepackage{stdtemplate}

\title{MAT354 Complex Analysis}
\author{Jonah Chen}
\date{}
\begin{document}
\maketitle
\sffamily

\section{Rational Functions}
\subsection{Classification of rational functions of order 2}
(up to fractional linear transformations of the source and target):
\begin{enumerate}
    \item One double pole $\beta$
    \item Two distinct poles $a,b$
\end{enumerate}

In case 1: Make a fractional linear transformation to move $\beta$ to $\infty$
\begin{equation}
    z=\beta+\frac{1}{\zeta}
\end{equation}
We set a rational function with double pole at $\infty$, i.e. a polynomial of degree $2$
\begin{align}
    w&=az^2+bz+c\\
    &=a\left(z+\frac{b}{2a}\right)^2-\frac{b^2}{4a}+c\\
\end{align}
Making a change of coordinates in the source and the target
\begin{align}
    w_1&=w+\frac{b^2}{4a}-c\\
    z_1&=z+\frac{b}{2a}\\
\end{align}
so we have $w_1=z_1^2$

In case 2: Make a fractional linear transformation to move $a,b$ to $0,\infty$.
\begin{align}
    w=\frac{z-b}{z-a}
\end{align}

Rational function of order 2 with poles at $0,\infty$ can be written $w=Az+B+\frac{C}{z}$. Make the coefficients of $z$ and $1/z$ equal by $z_1=\sqrt{\frac{A}{C}}z$ and $w_1=\frac{1}{A}(w-B)$ then $w=z+\frac{1}{z}$.

\subsection{Rational functions of order 1}
Fractional linear transformation \begin{align}
    w=S(z)=\frac{az+b}{cz+d}, ad-bc\neq0
\end{align}
Note that $S(\infty)=a/c$ and $S(-d/c)=\infty$.

We want to show that all fractional linear transformations can be written as a composition of translation, inversion, homothety

For $c=0$, $w=az+b$ which is a translation, homothety.

For $c\neq0$, \begin{equation}
    \frac{az+b}{cz+d}=\frac{\frac{a}{c}(z+d/c)+b+\frac{bc-ad}{c^2}}{z+d/c}=\frac{a}{c}+\frac{bc-ad}{c^2}\frac{1}{z+d/c}
\end{equation}
This is a composition of \begin{enumerate}
    \item translation: $z_1=z+d/c$
    \item inversion: $z_2=1/z_1$
    \item homethety: $z_3=\frac{bc-ad}{c^2}\cdot z_2$
    \item translation: $z_4=z_3+a/c$
\end{enumerate}

\begin{theorem}
    Given any 3 distinct points $z_2,z_3,z_4$, $\exists!$ fractional linear transformation $S:z_2,z_3,z_4\mapsto 1,0,\infty$
    \begin{proof}
        \begin{equation}
            S(z)=\begin{cases}
                \frac{z-z_3}{z-z_4}\Big/\frac{z_2-z_3}{z_2-z_4} & \text{otherwise}\\
                \frac{z-z_3}{z-z_4} & \text{if }z_2=\infty\\
                \frac{z_2-z_4}{z-z_4} & \text{if }z_3=\infty\\
                \frac{z-z_3}{z_2-z_3} & \text{if }z_4=\infty
            \end{cases}
        \end{equation}
        Suppose also $T:z_2,z_3,z_4\mapsto 1,0,\infty$. Consider $ST\inv:1,0,\infty\mapsto 1,0,\infty$. $ST\inv$ is also a fractional linear transformation $\frac{az_b}{cz+d}$

        Given any pair of circles/lines
    \end{proof}
\end{theorem}
\begin{definition}[Cross ratio]
    \begin{equation}
        (z_1:z_2:z_3:z_4)=S(z_1)
    \end{equation}
    is the cross ratio of $z_1,z_2,z_3,z_4$.
\end{definition}
\begin{theorem}
    \begin{enumerate}
        \item If $z_1,z_2,z_3,z_4$ are distinct points, and $T$ is a fractional linear transformation, then\begin{equation}
            (z_1:z_2:z_3:z_4)=(Tz_1:Tz_2:Tz_3:Tz_4)
        \end{equation}
        \item $(z_1:z_2:z_3:z_4)$ is real if and only if $z_1,z_2,z_3,z_4$ lie on a circle or a line.
    \end{enumerate}
    \begin{proof}
        \begin{enumerate}
            \item Let $Sz=(z:z_2:z_3:z_4)$. Then, $ST\inv:Tz_2,Tz_3,Tz_4\mapsto1,0,\infty$. Then, $(Tz_1:Tz_2:Tz_3:Tz_4)$ is by definition equal to $Tz_1$ under the fractional linear transformation that takes $Tz_2,Tz_3,Tz_4$ to $1,0,\infty$, which is precisely $ST\inv$. So, $(Tz_1:Tz_2:Tz_3:Tz_4)=ST\inv(Tz_1)=Sz_1=(z_1:z_2:z_3:z_4)$.
            \item First, we show the image of the real axis under fractional linear transformation $T\inv$ is either a circle of line.
            
            $w=T\inv(z)$ for $z\in\R$, we want to see that $w$ satisfies the equation of a circle or line. 
            
            We are interested in all $w$ such that $z=Tw=\frac{aw+b}{cw+d}$ is real. If $z\in\R$, then $Tw=\overline{Tw}$ and \begin{align}
                \frac{aw+b}{cw+d}=\frac{\bar a\bar w+\bar b}{\bar c\bar w+\bar d}\\
                (aw+b)(\bar c\bar w+\bar d)=(cw+d)(\bar a\bar w+\bar b)\\
                (\underbrace{a\bar c-\bar ac}_\text{imaginary})|w|^2+\underbrace{(a\bar d-\bar bc)w+(b\bar c-\bar ad)}_\text{imaginary}+\underbrace{b\bar d-\bar bd}_\text{imaginary}=0
            \end{align}
            If $a\bar c-\bar ac\neq 0$, then this is an equation of a circle. If $a\bar c-\bar ac=0$, then this is an equation of a line.

            Next, $Sz=(z:z_2:z_3:z_4)$ is real on the image of the real axis under $S\inv$ and nowhere else. $S\inv:1,0,\infty\mapsto z_2,z_3,z_4$
        \end{enumerate}
    \end{proof}
\end{theorem}

Fractional linear transformations $T$ takes the set of all circles and lines in the complex plane to itself. 

Given any pair of circles/lines, there is a fractional linear transformation taking one to the other.


\begin{example}
    Fractional linear transformation that takes the upper half plane $H^+$ to the unit disk $D$ and the real axis to the unit circle.

    We will take $i$ to $0$, so the numerator should be $z-i$. $w=\frac{z-i}{z+i}:i\mapsto0,0\mapsto-1,\infty\mapsto1,1\mapsto-i$
\end{example}

\section{Holomorphic Functions}
\begin{itemize}
    \item $f(z)$ complex valued functions in an open set $\Omega\subset\C$ or $\Omega\subset\C\cup\{\infty\}$
    \item $f$ is holomorphic if $\lim_{h\to0}\frac{f(z+h)-f(z)}{h}$ exists. i.e. for some $c\in\C, f(z+h)-f(z)=ch+\varphi(h)h$ where $\varphi(h)\in o(h)$.
    \item This is similar to the definition of the derivative from an open set in the plane to an open set in the plane. (writing $z=x+iy,f(z)=u+iv,c=a+ib,h=\xi+i\eta$ and $f:(x,y)\mapsto(u,v)$)
    \item The derivative at $z$ takes\begin{align}
        h&\mapsto ch\\
        \begin{pmatrix}
            \xi\\
            \eta
        \end{pmatrix}&\mapsto\begin{pmatrix}
            a & -b\\
            b & a
        \end{pmatrix}\begin{pmatrix}
            \xi\\
            \eta
        \end{pmatrix}
    \end{align}
    The matrix $\begin{pmatrix}
        a & -b\\
        b & a
    \end{pmatrix}=\begin{pmatrix}
        \partialderivative{u}{x} & \partialderivative{u}{y}\\
        \partialderivative{v}{x} & \partialderivative{v}{y}
    \end{pmatrix}$
    \item For a function to be holomorphic, it requires an additional constraint than being simply differentiable. $\partialderivative{u}{x}=\partialderivative{v}{y}$ and $\partialderivative{u}{y}=-\partialderivative{v}{x}$. This is the Cauchy-Riemann equations. Or, $\partialderivative{f}{x}+i\partialderivative{f}{y}=0$.
    \item The derivative at $z$ is a linear transformation $h\mapsto ch$.
    \item The jacobian determinant is $a^2+b^2=|f'(z)|^2$.
    \item Consider $f(x,y)$ differentiable, but complex valued. The differential $\dd f=\partialderivative{f}{x}\dd x+\partialderivative{f}{y}\dd y$. For example, $z=x+iy$ or $\bar z=x-iy$. Then, $\dd z=\dd x+i\dd y$ and $\dd\bar z=\dd x-i\dd y$.
    \item Then we have $\dd x=\frac{1}{2}(\dd z+\dd\bar z)$ and $\dd y=\frac{1}{2i}(\dd z-\dd\bar z)$. Then, \begin{align}
        \dd f=\frac{1}{2}\left(\partialderivative{f}{x}-i\partialderivative{f}{y}\right)\dd z+\frac{1}{2}\left(\partialderivative{f}{x}+i\partialderivative{f}{y}\right)\dd\bar z
    \end{align}
    \item So, we \redbf{define}\begin{align}
        \partialderivative{f}{z}=\frac{1}{2}\left(\partialderivative{f}{x}-i\partialderivative{f}{y}\right)\\
        \partialderivative{f}{\bar z}=\frac{1}{2}\left(\partialderivative{f}{x}+i\partialderivative{f}{y}\right)
    \end{align}
    \item Thus, we can write the 1-form $\dd f=\partialderivative{f}{z}\dd z+\partialderivative{f}{\bar z}\dd\bar z$.
    \item $\partialderivative{}{z}$ and $\partialderivative{}{\bar z}$ are defined as the dual basis for $\dd z,\dd\bar z$.
    \item We can rewrite the Cauchy-Riemann equations as $\partialderivative{f}{\bar z}=0$. This means for holomorphic functions, it's \redbf{only} a function of $z$, \redbf{not} $\bar z$.
    \begin{definition}[Harmonic Function]
        $f(x,y)$ is a \textbf{harmonic function} if $f\in C^2$ and $\Delta f=0$, or $\frac{\partial^2f}{\partial z\partial\bar z}=0$. (laplace equation) 
    \end{definition}
    \item We will see that holomorphic functions are harmonic. (but we need to first show we can differentiate holomorphic functions twice) So, the real and imaginary parts of holomorphic functions are also harmonic.
    \item Remark: $\partialderivative{f}{\bar z}=0\iff\partialderivative{\bar f}{z}=0$. Why? Consider $f=u+iv,\bar f=u-iv$.
    \begin{align}
        \partialderivative{f}{\bar z}&=\frac{1}{2}\left(\partialderivative{f}{x}+i\partialderivative{f}{y}\right)\\
        \overline{\partialderivative{f}{\bar z}}&=\frac{1}{2}\left(\partialderivative{\bar f}{x}-i\partialderivative{\bar f}{y}\right)=\partialderivative{\bar f}{z}\\
    \end{align}
    \begin{lemma}
        If $f(z)$ is holomorphic in a connected open set $\Omega$ and $f'(z)=0$ in $\Omega$, then $f$ is constant.
        \begin{proof}
            \begin{equation}
                \dd f=\underbrace{\partialderivative{f}{z}}_{0}\dd z+\underbrace{\partialderivative{f}{\bar z}}_{0\text{ holomorphic}}\dd\bar z=0
            \end{equation}
        \end{proof}
    \end{lemma}
    \begin{proposition}
        Given $f(z)$ is holomorphic in a connected open set $\Omega$, then
        \begin{enumerate}
            \item If $|f(z)$ is constant, then $f(z)$ is constant.
            \item If $\Re(f(z))$ is constant, then $f(z)$ is real.
        \end{enumerate}
        \begin{proof}
            \begin{enumerate}
                \item $|f(z)|^2=f(z)\overline{f(z)}$ is constant, so \begin{align}
                    0=\partialderivative{|f|^2}{z}=\partialderivative{f}{z}\bar f+f\cancel{\partialderivative{\bar f}{z}}=\partialderivative{f}{z}\bar z
                \end{align}
                so either $\bar f=0$ so $f=0$ thus $f$ is constant or $\partialderivative{f}{z}=0$ so $f$ is constant.
                \item $\Re(f)=f+\bar f$ is constant, so \begin{align}
                    0=\partialderivative{(f+\bar f)}{z}=\partialderivative{f}{z}+\cancel{\partialderivative{\bar f}{z}}=\partialderivative{f}{z}
                \end{align}
                so $\partialderivative{f}{z}=0$ and $f$ is constant.
            \end{enumerate}
        \end{proof}
    \end{proposition}
\end{itemize}
\subsection{Mapping Properties}
Suppose $f$ is holomorphic at some point $z_0.$ The \bluebf{tangent mapping} of $f$ at $z_0$ is \begin{equation}
    w=f(z_0)+f'(z_0)(z-z_0),
\end{equation}
if $f'(z)\neq 0, $ then the tangent mapping preserves angles and their orientation. \begin{definition}[Conformal Mapping]
    A mapping $f$ is \textbf{conformal} if $f$ is holomorphic and $f'(z_0)\neq 0$. i.e. if $f$ preserves angles and orientation.
\end{definition}
\begin{lemma}
    A $\R$-linear transformation $\C\to\C$ which preserves angles is of the form either $w=cz$ or $w=c\bar z.$
\end{lemma}
Consider $w=f(z)$ in a connected open set $\Omega.$ If $f$ is treated as a function from $\R^2\to\R^2$ has $\det f'\neq0$ in $\Omega.$

If $f$ preserves angles at every point in $\Omega,$ then $\partialderivative{f}{z}=0$ or $\partialderivative{f}{\bar z}=0.$ They cannot be both zero at the same point, as otherwise $\det f'=0$ at that point. As $f\in C^1,$ the partial derivatives are continuous. This means $\{z\in\Omega|\partialderivative{f}{z}=0\},\{z\in\Omega|\partialderivative{f}{\bar z}=0\}$ are disjoint sets, and their union is $\Omega.$ Since $\Omega$ is connected, one of them must be empty.

So, either $\partialderivative{f}{\bar z}=0$ throughout $\Omega\implies f$ is holomorphic, or $\partialderivative{f}{\bar z}=0$ throughout $\Omega\implies f$ is anti-holomorphic.

\begin{theorem}
    $f$ preserves angles at every point in $\Omega\iff f$ is either holomorphic or anti-holomorphic in $\Omega.$
\end{theorem}

\begin{theorem}[Inverse Function]
    Suppose $f$ is holomorphic in a neighborhood of $z_0$ and $f'(z_0)\neq0.$ Then there are neighborhoods $U$ of $z_0$ and $V$ of $w_0=f(z_0)$ such that $f$ maps $U$ \bluebf{onto} $V,$ with an inverse $z=g(w)$ which is holomorphic in $V.$ And,\begin{align}
        g'(w)=\frac{1}{f'(z)}.
    \end{align}
    \begin{proof}[Proof (to be completed later)] We will use the fact that partial derivatives of holomorphic functions are continuous, which we will prove later.

    If $f'(z)=\begin{pmatrix}
        a&-b\\
        b&a
    \end{pmatrix}$
    and $g'$ is the inverse, then $g'(w)=\frac{1}{a^2+b^2}\begin{pmatrix}
        a&b\\
        -b&a
    \end{pmatrix},$ so $g$ satisfies the cauchy riemann equations and $g$ is holomorphic.
    \end{proof}
\end{theorem}

\section{Power Series}
\begin{itemize}
    \item A complex power series $f(w)=\sum_{n=0}^\infty a_nw^n.$ Note that $w$ is not a complex number, it's just a symbol. Complex power series means $a_n\in\C.$
    \item Suppose we have another power series $g(z)=\sum_{p=0}^\infty b_pz^p.$ We want to compose \begin{align}
        (f\circ g)(z)=a_0+a_1(b_0+b_1z+\cdots)+a_2(b_0+b_1z+\cdots)^2+\cdots
    \end{align} 
    \item First we need to ask if this even make sense? The answer is yes if $b_0=0.$ However, in calculus every formal power series is the taylor series of some $C^\infty$ functions, which can be composed. So why do we have this restriction?
    \item Consider taylor series of $f(g(z))$ at $z=z_0.$ Let $w_0=g(z_0)$ and the taylor series at $w_0$ is \begin{align}
        f(w)=\sum_{n=0}^\infty a_n(w-w_0)^n.
    \end{align}
    Then we replace $w$ with the taylor series for $g$ at $z_0,$ with $b_0=w_0$ so these does not have constant term.
\end{itemize}
\begin{definition}[Formal Derivative]
    We define $f(0)=a_0$ and the \bluebf{formal derivative} of $f(w)$ as \begin{align}
        f'(w)=\sum_{n=1}^\infty na_nw^{n-1}.
    \end{align}
\end{definition}
\begin{theorem}[Formal inverse function]
    Given formal power series $f(w)=\sum_{n=0}^\infty a_nw^n.$ There is a power series $g(z)=\sum_{p=0}^\infty b_pz^p$ such that $b_0=0$ and $f\circ g=\mathrm{id}$ where $\mathrm{id}(z)=z$ \redbf{iff} $f(0)=0,f'(0)\neq0.$ In that case $g$ is uniquely determined by $f$ and $g\circ f=\mathrm{id}$ also.
    \begin{proof}[Proof by method of undetermined coefficients]
        We are trying to solve \begin{align}
            a_0+a_1(b_1z+b_2z^2+\cdots)+a_2(b_1z+b_2z^2+\cdots)^2+\cdots=z.
        \end{align}
        We know right away that $a_0=0$ and $a_1b_1=1.$ so we know that $a_0=0$ and $a_1\neq=$ are necessary conditions. Conversely, they are sufficient to solve for \bluebf{unique} coefficients of $g$.

        The coefficient of $z^n$ on the LHS is the same as the coefficient of $z^n$ in \begin{align}
            \cancel{a_0}+a_1g(z)+\cdots+a_ng(z)^n=a_1b_n+P(a_2,\dots,a_n,b_1,\dots,b_{n-1}).
        \end{align}
        And $b_1=1/a_1,$ thus $b_n$ can be calculated recursively.

        Since $g(0)=0$ and $g'(0)\neq0,$ there is a unique formal power series $f_1(w)$ s.t. $g\circ f_1=\mathrm{id}.$\begin{align}
            f_1=\mathrm{id}\circ f_1=(f\circ g)\circ f_1=f\circ(g\circ f_1)=f
        \end{align}
    \end{proof}
\end{theorem}
\begin{proposition}
    If $f=\sum_{n=0}^\infty a_nw^n$ and $g=\sum_{p=0}^\infty b_pw^p$ are convergent power series, then $f\circ g$ is also convergent. In fact, take $r>0$ s.t. $\sum_{p=1}^\infty|b_p|r^p<R(f)$ the radius convergence of $f.$ Then,\begin{enumerate}[label=(\arabic*)]
        \item $R(f\circ g)\geq r$
        \item If $|z|<r$ then $|g(z)<R(f).$
        \item $f(g(z))=(f\circ g)(z)$ (by rearrangement of absolute convergent series) where RHS is formal power series composition and LHS is substituting the value of $g(z)$ into $f.$
    \end{enumerate}
    \begin{proof}[Proof of (1)]
        \begin{align}
            \sum_{n=0}^\infty|a_n|\left(\sum_{p=1}^\infty|b_p|r^p\right)^n=:\sum_{k=0}^\infty\gamma_kr^k<\infty
        \end{align}
        Say $(f\circ g)(z)=\sum c_kz^k.$ By triangle inequality, $|c_k|\leq\gamma_k.$ As $\sum\gamma_kr^k<\infty,$ then $\sum c_k\gamma^k$ is convergent.
    \end{proof}
\end{proposition}
\begin{theorem}[Reciprocal] 
    If $f(z)=\sum_{n=0}^\infty a_nz^n$ and $a_0\neq0$ then there is an unique power series $g(z)$ s.t. $f(z)=g(z)=1.$ If $f$ has a positive radius of convergence, then so does $z.$
    \begin{proof}
        As $a_0\neq0,$ then WLOG $a_0=1.$ Write $f(z)=1-h(z)$ then \begin{align}
            f(z)\inv=(1-h(z))\inv=1+\sum_{n=1}^\infty w^n\quad \text{where }w=h(z).
        \end{align}
    \end{proof}
\end{theorem}
\begin{theorem}[Inverse function for convergent power series]
    In the previous statement, if $f(w)$ has a positive radius of convergence, then so does $g(z).$
    \begin{proof}
        By direct estimate OR follows from inverse function theorem for holomorphic functions once we know holomorphic function has infinite taylor series that converges.
    \end{proof}
\end{theorem}

\subsection{Logarithmic Function}
\begin{itemize}
    \item The principal branch of $\log z$ is defined on the largest simply connected set that does not contain zero, which we will choose $\C\setminus(-\infty,0].$ In this domain, there is a unique value of $\arg z\in(-\pi,\pi),$ we will call it $\mathrm{Arg}(z).$
    \item We can show that this is continuous by showing it is continuous on $S'\setminus\{-1\}.$ We can show this by its the fact its inverse $z=e^{i\theta}$ is continuous on $[-(\pi-\epsilon),\pi+\epsilon]$ hence the it's the inverse of an bijection on compact hausdorff space.
    \item The principal branch of $\log z$ is defined as $\log|z|+i\mathrm{Arg}\,z,$ which is continuous on its entire domain $\C\setminus(-\infty,0].$ Note that this is equal to the real logarithm if $z\in\R.$
    \begin{proposition}
        The power series $f(z)=\sum_{n=1}^\infty(-1)^{n+1}\frac{z^n}{n}$ converges if $|z|<1$ and the sum is equal to the principal branch of $\log(1+z).$ \begin{proof}
            The power series $f(z)$ and $g(w)=\sum_{n=1}^\infty\frac{w^n}{n!}=e^w-1$ are inverses. The proof is by MAT157 since the coefficients here are all real with $g(f(z))=z$ when $|z|<1$.

            We also know that $e^{f(z)}=1+z$ and it's the principal branch because $f(0)=\log 1=0$
        \end{proof}
    \end{proposition}
    \begin{definition}[Power]
        \begin{align}
            z^\alpha=e^{\alpha\log z}
        \end{align}
        where $\alpha\in\C,z\neq0.$ Note that for fixed $\alpha,\: z^\alpha$ is a many-valued function of $z.$ This has a branch in any \bluebf{domain} (connected open subset of $\C$) where $\log$ has a branch. \redbf{Any} branch of $\log z$ in $\Omega$ defines a branch of $z^\alpha.$
    \end{definition}
    \begin{itemize}
        \item e.g. The \bluebf{binomial series} $(1+z)^\alpha=e^{\alpha\log(1+z)}$ and its power series expansion in $|z|<1$ is $\sum\binom{\alpha}{n}z^n.$
    \end{itemize}
\end{itemize}



\end{document}
