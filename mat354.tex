\documentclass[a4paper,12pt]{article}
\usepackage{stdtemplate}

\title{MAT354 Complex Analysis}
\author{Jonah Chen}
\date{}
\begin{document}
\maketitle
\sffamily

\section{Rational Functions}
\subsection{Classification of rational functions of order 2}
(up to fractional linear transformations of the source and target):
\begin{enumerate}
    \item One double pole $\beta$
    \item Two distinct poles $a,b$
\end{enumerate}

In case 1: Make a fractional linear transformation to move $\beta$ to $\infty$
\begin{equation}
    z=\beta+\frac{1}{\zeta}
\end{equation}
We set a rational function with double pole at $\infty$, i.e. a polynomial of degree $2$
\begin{align}
    w&=az^2+bz+c\\
    &=a\left(z+\frac{b}{2a}\right)^2-\frac{b^2}{4a}+c\\
\end{align}
Making a change of coordinates in the source and the target
\begin{align}
    w_1&=w+\frac{b^2}{4a}-c\\
    z_1&=z+\frac{b}{2a}\\
\end{align}
so we have $w_1=z_1^2$

In case 2: Make a fractional linear transformation to move $a,b$ to $0,\infty$.
\begin{align}
    w=\frac{z-b}{z-a}
\end{align}

Rational function of order 2 with poles at $0,\infty$ can be written $w=Az+B+\frac{C}{z}$. Make the coefficients of $z$ and $1/z$ equal by $z_1=\sqrt{\frac{A}{C}}z$ and $w_1=\frac{1}{A}(w-B)$ then $w=z+\frac{1}{z}$.

\subsection{Rational functions of order 1}
Fractional linear transformation \begin{align}
    w=S(z)=\frac{az+b}{cz+d}, ad-bc\neq0
\end{align}
Note that $S(\infty)=a/c$ and $S(-d/c)=\infty$.

We want to show that all fractional linear transformations can be written as a composition of translation, inversion, homothety

For $c=0$, $w=az+b$ which is a translation, homothety.

For $c\neq0$, \begin{equation}
    \frac{az+b}{cz+d}=\frac{\frac{a}{c}(z+d/c)+b+\frac{bc-ad}{c^2}}{z+d/c}=\frac{a}{c}+\frac{bc-ad}{c^2}\frac{1}{z+d/c}
\end{equation}
This is a composition of \begin{enumerate}
    \item translation: $z_1=z+d/c$
    \item inversion: $z_2=1/z_1$
    \item homethety: $z_3=\frac{bc-ad}{c^2}\cdot z_2$
    \item translation: $z_4=z_3+a/c$
\end{enumerate}

\begin{theorem}
    Given any 3 distinct points $z_2,z_3,z_4$, $\exists!$ fractional linear transformation $S:z_2,z_3,z_4\mapsto 1,0,\infty$
    \begin{proof}
        \begin{equation}
            S(z)=\begin{cases}
                \frac{z-z_3}{z-z_4}\Big/\frac{z_2-z_3}{z_2-z_4} & \text{otherwise}\\
                \frac{z-z_3}{z-z_4} & \text{if }z_2=\infty\\
                \frac{z_2-z_4}{z-z_4} & \text{if }z_3=\infty\\
                \frac{z-z_3}{z_2-z_3} & \text{if }z_4=\infty
            \end{cases}
        \end{equation}
        Suppose also $T:z_2,z_3,z_4\mapsto 1,0,\infty$. Consider $ST\inv:1,0,\infty\mapsto 1,0,\infty$. $ST\inv$ is also a fractional linear transformation $\frac{az_b}{cz+d}$

        Given any pair of circles/lines
    \end{proof}
\end{theorem}
\begin{definition}[Cross ratio]
    \begin{equation}
        (z_1:z_2:z_3:z_4)=S(z_1)
    \end{equation}
    is the cross ratio of $z_1,z_2,z_3,z_4$.
\end{definition}
\begin{theorem}
    \begin{enumerate}
        \item If $z_1,z_2,z_3,z_4$ are distinct points, and $T$ is a fractional linear transformation, then\begin{equation}
            (z_1:z_2:z_3:z_4)=(Tz_1:Tz_2:Tz_3:Tz_4)
        \end{equation}
        \item $(z_1:z_2:z_3:z_4)$ is real if and only if $z_1,z_2,z_3,z_4$ lie on a circle or a line.
    \end{enumerate}
    \begin{proof}
        \begin{enumerate}
            \item Let $Sz=(z:z_2:z_3:z_4)$. Then, $ST\inv:Tz_2,Tz_3,Tz_4\mapsto1,0,\infty$. Then, $(Tz_1:Tz_2:Tz_3:Tz_4)$ is by definition equal to $Tz_1$ under the fractional linear transformation that takes $Tz_2,Tz_3,Tz_4$ to $1,0,\infty$, which is precisely $ST\inv$. So, $(Tz_1:Tz_2:Tz_3:Tz_4)=ST\inv(Tz_1)=Sz_1=(z_1:z_2:z_3:z_4)$.
            \item First, we show the image of the real axis under fractional linear transformation $T\inv$ is either a circle of line.
            
            $w=T\inv(z)$ for $z\in\R$, we want to see that $w$ satisfies the equation of a circle or line. 
            
            We are interested in all $w$ such that $z=Tw=\frac{aw+b}{cw+d}$ is real. If $z\in\R$, then $Tw=\overline{Tw}$ and \begin{align}
                \frac{aw+b}{cw+d}=\frac{\bar a\bar w+\bar b}{\bar c\bar w+\bar d}\\
                (aw+b)(\bar c\bar w+\bar d)=(cw+d)(\bar a\bar w+\bar b)\\
                (\underbrace{a\bar c-\bar ac}_\text{imaginary})|w|^2+\underbrace{(a\bar d-\bar bc)w+(b\bar c-\bar ad)}_\text{imaginary}+\underbrace{b\bar d-\bar bd}_\text{imaginary}=0
            \end{align}
            If $a\bar c-\bar ac\neq 0$, then this is an equation of a circle. If $a\bar c-\bar ac=0$, then this is an equation of a line.

            Next, $Sz=(z:z_2:z_3:z_4)$ is real on the image of the real axis under $S\inv$ and nowhere else. $S\inv:1,0,\infty\mapsto z_2,z_3,z_4$
        \end{enumerate}
    \end{proof}
\end{theorem}

Fractional linear transformations $T$ takes the set of all circles and lines in the complex plane to itself. 

Given any pair of circles/lines, there is a fractional linear transformation taking one to the other.


\begin{example}
    Fractional linear transformation that takes the upper half plane $H^+$ to the unit disk $D$ and the real axis to the unit circle.

    We will take $i$ to $0$, so the numerator should be $z-i$. $w=\frac{z-i}{z+i}:i\mapsto0,0\mapsto-1,\infty\mapsto1,1\mapsto-i$
\end{example}

\section{Holomorphic Functions}
\begin{itemize}
    \item $f(z)$ complex valued functions in an open set $\Omega\subset\C$ or $\Omega\subset\C\cup\{\infty\}$
    \item $f$ is holomorphic if $\lim_{h\to0}\frac{f(z+h)-f(z)}{h}$ exists. i.e. for some $c\in\C, f(z+h)-f(z)=ch+\varphi(h)h$ where $\varphi(h)\in o(h)$.
    \item This is similar to the definition of the derivative from an open set in the plane to an open set in the plane. (writing $z=x+iy,f(z)=u+iv,c=a+ib,h=\xi+i\eta$ and $f:(x,y)\mapsto(u,v)$)
    \item The derivative at $z$ takes\begin{align}
        h&\mapsto ch\\
        \begin{pmatrix}
            \xi\\
            \eta
        \end{pmatrix}&\mapsto\begin{pmatrix}
            a & -b\\
            b & a
        \end{pmatrix}\begin{pmatrix}
            \xi\\
            \eta
        \end{pmatrix}
    \end{align}
    The matrix $\begin{pmatrix}
        a & -b\\
        b & a
    \end{pmatrix}=\begin{pmatrix}
        \partialderivative{u}{x} & \partialderivative{u}{y}\\
        \partialderivative{v}{x} & \partialderivative{v}{y}
    \end{pmatrix}$
    \item For a function to be holomorphic, it requires an additional constraint than being simply differentiable. $\partialderivative{u}{x}=\partialderivative{v}{y}$ and $\partialderivative{u}{y}=-\partialderivative{v}{x}$. This is the Cauchy-Riemann equations. Or, $\partialderivative{f}{x}+i\partialderivative{f}{y}=0$.
    \item The derivative at $z$ is a linear transformation $h\mapsto ch$.
    \item The jacobian determinant is $a^2+b^2=|f'(z)|^2$.
    \item Consider $f(x,y)$ differentiable, but complex valued. The differential $\dd f=\partialderivative{f}{x}\dd x+\partialderivative{f}{y}\dd y$. For example, $z=x+iy$ or $\bar z=x-iy$. Then, $\dd z=\dd x+i\dd y$ and $\dd\bar z=\dd x-i\dd y$.
    \item Then we have $\dd x=\frac{1}{2}(\dd z+\dd\bar z)$ and $\dd y=\frac{1}{2i}(\dd z-\dd\bar z)$. Then, \begin{align}
        \dd f=\frac{1}{2}\left(\partialderivative{f}{x}-i\partialderivative{f}{y}\right)\dd z+\frac{1}{2}\left(\partialderivative{f}{x}+i\partialderivative{f}{y}\right)\dd\bar z
    \end{align}
    \item So, we \redbf{define}\begin{align}
        \partialderivative{f}{z}=\frac{1}{2}\left(\partialderivative{f}{x}-i\partialderivative{f}{y}\right)\\
        \partialderivative{f}{\bar z}=\frac{1}{2}\left(\partialderivative{f}{x}+i\partialderivative{f}{y}\right)
    \end{align}
    \item Thus, we can write the 1-form $\dd f=\partialderivative{f}{z}\dd z+\partialderivative{f}{\bar z}\dd\bar z$.
    \item $\partialderivative{}{z}$ and $\partialderivative{}{\bar z}$ are defined as the dual basis for $\dd z,\dd\bar z$.
    \item We can rewrite the Cauchy-Riemann equations as $\partialderivative{f}{\bar z}=0$. This means for holomorphic functions, it's \redbf{only} a function of $z$, \redbf{not} $\bar z$.
    \begin{definition}[Harmonic Function]
        $f(x,y)$ is a \textbf{harmonic function} if $f\in C^2$ and $\Delta f=0$, or $\frac{\partial^2f}{\partial z\partial\bar z}=0$. (laplace equation) 
    \end{definition}
    \item We will see that holomorphic functions are harmonic. (but we need to first show we can differentiate holomorphic functions twice) So, the real and imaginary parts of holomorphic functions are also harmonic.
    \item Remark: $\partialderivative{f}{\bar z}=0\iff\partialderivative{\bar f}{z}=0$. Why? Consider $f=u+iv,\bar f=u-iv$.
    \begin{align}
        \partialderivative{f}{\bar z}&=\frac{1}{2}\left(\partialderivative{f}{x}+i\partialderivative{f}{y}\right)\\
        \overline{\partialderivative{f}{\bar z}}&=\frac{1}{2}\left(\partialderivative{\bar f}{x}-i\partialderivative{\bar f}{y}\right)=\partialderivative{\bar f}{z}\\
    \end{align}
    \begin{lemma}
        If $f(z)$ is holomorphic in a connected open set $\Omega$ and $f'(z)=0$ in $\Omega$, then $f$ is constant.
        \begin{proof}
            \begin{equation}
                \dd f=\underbrace{\partialderivative{f}{z}}_{0}\dd z+\underbrace{\partialderivative{f}{\bar z}}_{0\text{ holomorphic}}\dd\bar z=0
            \end{equation}
        \end{proof}
    \end{lemma}
    \begin{proposition}
        Given $f(z)$ is holomorphic in a connected open set $\Omega$, then
        \begin{enumerate}
            \item If $|f(z)$ is constant, then $f(z)$ is constant.
            \item If $\Re(f(z))$ is constant, then $f(z)$ is real.
        \end{enumerate}
        \begin{proof}
            \begin{enumerate}
                \item $|f(z)|^2=f(z)\overline{f(z)}$ is constant, so \begin{align}
                    0=\partialderivative{|f|^2}{z}=\partialderivative{f}{z}\bar f+f\cancel{\partialderivative{\bar f}{z}}=\partialderivative{f}{z}\bar z
                \end{align}
                so either $\bar f=0$ so $f=0$ thus $f$ is constant or $\partialderivative{f}{z}=0$ so $f$ is constant.
                \item $\Re(f)=f+\bar f$ is constant, so \begin{align}
                    0=\partialderivative{(f+\bar f)}{z}=\partialderivative{f}{z}+\cancel{\partialderivative{\bar f}{z}}=\partialderivative{f}{z}
                \end{align}
                so $\partialderivative{f}{z}=0$ and $f$ is constant.
            \end{enumerate}
        \end{proof}
    \end{proposition}
\end{itemize}
\subsection{Mapping Properties}
Suppose $f$ is holomorphic at some point $z_0.$ The \bluebf{tangent mapping} of $f$ at $z_0$ is \begin{equation}
    w=f(z_0)+f'(z_0)(z-z_0),
\end{equation}
if $f'(z)\neq 0, $ then the tangent mapping preserves angles and their orientation. \begin{definition}[Conformal Mapping]
    A mapping $f$ is \textbf{conformal} if $f$ is holomorphic and $f'(z_0)\neq 0$. i.e. if $f$ preserves angles and orientation.
\end{definition}
\begin{lemma}
    A $\R$-linear transformation $\C\to\C$ which preserves angles is of the form either $w=cz$ or $w=c\bar z.$
\end{lemma}
Consider $w=f(z)$ in a connected open set $\Omega.$ If $f$ is treated as a function from $\R^2\to\R^2$ has $\det f'\neq0$ in $\Omega.$

If $f$ preserves angles at every point in $\Omega,$ then $\partialderivative{f}{z}=0$ or $\partialderivative{f}{\bar z}=0.$ They cannot be both zero at the same point, as otherwise $\det f'=0$ at that point. As $f\in C^1,$ the partial derivatives are continuous. This means $\{z\in\Omega|\partialderivative{f}{z}=0\},\{z\in\Omega|\partialderivative{f}{\bar z}=0\}$ are disjoint sets, and their union is $\Omega.$ Since $\Omega$ is connected, one of them must be empty.

So, either $\partialderivative{f}{\bar z}=0$ throughout $\Omega\implies f$ is holomorphic, or $\partialderivative{f}{\bar z}=0$ throughout $\Omega\implies f$ is anti-holomorphic.

\begin{theorem}
    $f$ preserves angles at every point in $\Omega\iff f$ is either holomorphic or anti-holomorphic in $\Omega.$
\end{theorem}

\begin{theorem}[Inverse Function]
    Suppose $f$ is holomorphic in a neighborhood of $z_0$ and $f'(z_0)\neq0.$ Then there are neighborhoods $U$ of $z_0$ and $V$ of $w_0=f(z_0)$ such that $f$ maps $U$ \bluebf{onto} $V,$ with an inverse $z=g(w)$ which is holomorphic in $V.$ And,\begin{align}
        g'(w)=\frac{1}{f'(z)}.
    \end{align}
    \begin{proof}[Proof (to be completed later)] We will use the fact that partial derivatives of holomorphic functions are continuous, which we will prove later.

    If $f'(z)=\begin{pmatrix}
        a&-b\\
        b&a
    \end{pmatrix}$
    and $g'$ is the inverse, then $g'(w)=\frac{1}{a^2+b^2}\begin{pmatrix}
        a&b\\
        -b&a
    \end{pmatrix},$ so $g$ satisfies the cauchy riemann equations and $g$ is holomorphic.
    \end{proof}
\end{theorem}

\section{Power Series}
\begin{itemize}
    \item A complex power series $f(w)=\sum_{n=0}^\infty a_nw^n.$ Note that $w$ is not a complex number, it's just a symbol. Complex power series means $a_n\in\C.$
    \item Suppose we have another power series $g(z)=\sum_{p=0}^\infty b_pz^p.$ We want to compose \begin{align}
        (f\circ g)(z)=a_0+a_1(b_0+b_1z+\cdots)+a_2(b_0+b_1z+\cdots)^2+\cdots
    \end{align} 
    \item First we need to ask if this even make sense? The answer is yes if $b_0=0.$ However, in calculus every formal power series is the taylor series of some $C^\infty$ functions, which can be composed. So why do we have this restriction?
    \item Consider taylor series of $f(g(z))$ at $z=z_0.$ Let $w_0=g(z_0)$ and the taylor series at $w_0$ is \begin{align}
        f(w)=\sum_{n=0}^\infty a_n(w-w_0)^n.
    \end{align}
    Then we replace $w$ with the taylor series for $g$ at $z_0,$ with $b_0=w_0$ so these does not have constant term.
\end{itemize}
\begin{definition}[Formal Derivative]
    We define $f(0)=a_0$ and the \bluebf{formal derivative} of $f(w)$ as \begin{align}
        f'(w)=\sum_{n=1}^\infty na_nw^{n-1}.
    \end{align}
\end{definition}
\begin{theorem}[Formal inverse function]
    Given formal power series $f(w)=\sum_{n=0}^\infty a_nw^n.$ There is a power series $g(z)=\sum_{p=0}^\infty b_pz^p$ such that $b_0=0$ and $f\circ g=\mathrm{id}$ where $\mathrm{id}(z)=z$ \redbf{iff} $f(0)=0,f'(0)\neq0.$ In that case $g$ is uniquely determined by $f$ and $g\circ f=\mathrm{id}$ also.
    \begin{proof}[Proof by method of undetermined coefficients]
        We are trying to solve \begin{align}
            a_0+a_1(b_1z+b_2z^2+\cdots)+a_2(b_1z+b_2z^2+\cdots)^2+\cdots=z.
        \end{align}
        We know right away that $a_0=0$ and $a_1b_1=1.$ so we know that $a_0=0$ and $a_1\neq=$ are necessary conditions. Conversely, they are sufficient to solve for \bluebf{unique} coefficients of $g$.

        The coefficient of $z^n$ on the LHS is the same as the coefficient of $z^n$ in \begin{align}
            \cancel{a_0}+a_1g(z)+\cdots+a_ng(z)^n=a_1b_n+P(a_2,\dots,a_n,b_1,\dots,b_{n-1}).
        \end{align}
        And $b_1=1/a_1,$ thus $b_n$ can be calculated recursively.

        Since $g(0)=0$ and $g'(0)\neq0,$ there is a unique formal power series $f_1(w)$ s.t. $g\circ f_1=\mathrm{id}.$\begin{align}
            f_1=\mathrm{id}\circ f_1=(f\circ g)\circ f_1=f\circ(g\circ f_1)=f
        \end{align}
    \end{proof}
\end{theorem}
\begin{proposition}
    If $f=\sum_{n=0}^\infty a_nw^n$ and $g=\sum_{p=0}^\infty b_pw^p$ are convergent power series, then $f\circ g$ is also convergent. In fact, take $r>0$ s.t. $\sum_{p=1}^\infty|b_p|r^p<R(f)$ the radius convergence of $f.$ Then,\begin{enumerate}[label=(\arabic*)]
        \item $R(f\circ g)\geq r$
        \item If $|z|<r$ then $|g(z)<R(f).$
        \item $f(g(z))=(f\circ g)(z)$ (by rearrangement of absolute convergent series) where RHS is formal power series composition and LHS is substituting the value of $g(z)$ into $f.$
    \end{enumerate}
    \begin{proof}[Proof of (1)]
        \begin{align}
            \sum_{n=0}^\infty|a_n|\left(\sum_{p=1}^\infty|b_p|r^p\right)^n=:\sum_{k=0}^\infty\gamma_kr^k<\infty
        \end{align}
        Say $(f\circ g)(z)=\sum c_kz^k.$ By triangle inequality, $|c_k|\leq\gamma_k.$ As $\sum\gamma_kr^k<\infty,$ then $\sum c_k\gamma^k$ is convergent.
    \end{proof}
\end{proposition}
\begin{theorem}[Reciprocal] 
    If $f(z)=\sum_{n=0}^\infty a_nz^n$ and $a_0\neq0$ then there is an unique power series $g(z)$ s.t. $f(z)=g(z)=1.$ If $f$ has a positive radius of convergence, then so does $z.$
    \begin{proof}
        As $a_0\neq0,$ then WLOG $a_0=1.$ Write $f(z)=1-h(z)$ then \begin{align}
            f(z)\inv=(1-h(z))\inv=1+\sum_{n=1}^\infty w^n\quad \text{where }w=h(z).
        \end{align}
    \end{proof}
\end{theorem}
\begin{theorem}[Inverse function for convergent power series]
    In the previous statement, if $f(w)$ has a positive radius of convergence, then so does $g(z).$
    \begin{proof}
        By direct estimate OR follows from inverse function theorem for holomorphic functions once we know holomorphic function has infinite taylor series that converges.
    \end{proof}
\end{theorem}

\subsection{Logarithmic Function}
\begin{itemize}
    \item The principal branch of $\log z$ is defined on the largest simply connected set that does not contain zero, which we will choose $\C\setminus(-\infty,0].$ In this domain, there is a unique value of $\arg z\in(-\pi,\pi),$ we will call it $\mathrm{Arg}(z).$
    \item We can show that this is continuous by showing it is continuous on $S'\setminus\{-1\}.$ We can show this by its the fact its inverse $z=e^{i\theta}$ is continuous on $[-(\pi-\epsilon),\pi+\epsilon]$ hence the it's the inverse of an bijection on compact hausdorff space.
    \item The principal branch of $\log z$ is defined as $\log|z|+i\mathrm{Arg}\,z,$ which is continuous on its entire domain $\C\setminus(-\infty,0].$ Note that this is equal to the real logarithm if $z\in\R.$
    \begin{proposition}
        The power series $f(z)=\sum_{n=1}^\infty(-1)^{n+1}\frac{z^n}{n}$ converges if $|z|<1$ and the sum is equal to the principal branch of $\log(1+z).$ \begin{proof}
            The power series $f(z)$ and $g(w)=\sum_{n=1}^\infty\frac{w^n}{n!}=e^w-1$ are inverses. The proof is by MAT157 since the coefficients here are all real with $g(f(z))=z$ when $|z|<1$.

            We also know that $e^{f(z)}=1+z$ and it's the principal branch because $f(0)=\log 1=0$
        \end{proof}
    \end{proposition}
    \begin{definition}[Power]
        \begin{align}
            z^\alpha=e^{\alpha\log z}
        \end{align}
        where $\alpha\in\C,z\neq0.$ Note that for fixed $\alpha,\: z^\alpha$ is a many-valued function of $z.$ This has a branch in any \bluebf{domain} (connected open subset of $\C$) where $\log$ has a branch. \redbf{Any} branch of $\log z$ in $\Omega$ defines a branch of $z^\alpha.$
    \end{definition}
    \begin{itemize}
        \item e.g. The \bluebf{binomial series} $(1+z)^\alpha=e^{\alpha\log(1+z)}$ and its power series expansion in $|z|<1$ is $\sum\binom{\alpha}{n}z^n.$
    \end{itemize}
\end{itemize}

Mapping Properties of Holomorphic Functions
\begin{itemize}
    \item $w=z^\alpha$ for real, positive $\alpha$ maps angles $\theta$ to an angle $\alpha\theta.$
    \item In general $z^\alpha$ is not 1-1 if $\alpha\neq1,$ and is multi-valued if $\alpha$ is fractional.
    \item Often, we will use a branched covering (mapping $X\to\C$) so we can have a single valued branch. Consider the multi-valued function $w=z^{1/2}.$ Consider \begin{align}
        X=\{(z,w)\in\C^2|z=w^2\}
    \end{align}
    $X$ is a manifold (it is a graph of a continuous function) with local coordinate $w.$
    \item This multi-valued function $w=z^{1/2}$ lifts to a single valued $(w,z)\mapsto w$ by the covering surface $X.$ $X$ is an example of a \bluebf{Riemann surface}.
\end{itemize}

Consider a mapping that takes the upper half plane $H^+=\{z\in\C|\Im(z)>0\},$ and consider the mapping that takes $H^+\to D=\{z\in\C|z|<1\},$ We can use a fractional linear transformation \begin{align}
    w=\frac{z-i}{z+i}
\end{align}
this takes $i\mapsto0.$ We also know that it maps $\R$ to $S^1$ as we can pick three points $0,1,\infty\in\R$ and we know $0\mapsto-1.$ We know this preserves orientations so $1\mapsto-i$ and $\infty\mapsto1.$

Now, we want to find a conformal mapping of a circular wedge onto $D$ or $H^+.$

If  circular wedge is formed by two circles intersecting in $a$ and $b,$ first use a fractional linear transformation $\zeta=\frac{z-a}{z-b}$ to map $a\mapsto0$ and $b\mapsto\infty.$ This takes the two circles into rays. Then, we can rotate the region by multiplying a complex number $e^{i\theta}$ and then change the angle by taking $w=e^{i\theta}\zeta^\alpha$ for some power $\alpha.$

In the case they are degenerate, and only intersect at $a,$ we take $\zeta=\frac{1}{z-a}$ which leads to two parallel lines. Then we can rotate and stretch it so that they become the real line and the line $\Im(z)=\pi,$ then $\exp$ will map it to the upper half plane.

\textbf{Exercise:} Find a conformal mapping that takes the complement of the line segment to the interior (or exterior) of the unit disk. We will apply \begin{equation}
    z_1=\frac{z-1}{z+1}
\end{equation}
will map the interval $[-1,1]$ to $(-\infty,0].$ Then we can apply \begin{align}
    z_2=z_1^{1/2}
\end{align}
This maps the set to the right half plane. Finally, \begin{align}
    w=\frac{z_2-1}{z_2+1}=z-\sqrt{z^2-1}
\end{align}
will map the right half plane into the interior of the unit disk (flipping the fraction maps to exterior). Check which branch of square root we need to use? Finally, show that $z=\frac{1}{2}(w+1/w)$

\subsection{Mapping Properties of $\exp$ and $\log$}
\begin{itemize}
    \item We know that $w=e^z$ is periodic with period $2\pi i,$\begin{align}
        e^z=e^xe^{iy}\\
        &=e^x(\cos y+i\sin y)\\
    \end{align}
    \item The exponential maps a vertical line to circle about $0,$ a horizontal line to a ray through $0,$ and any other line to a logarithmic spiral.
    \item The exponential is not injective. To make it single-valued, we need to restrict its domain. The image of $e^z$ on $a<\Im z<b$ is a wedge in the complex plane $a<\arg w<b.$
    \item The logarithm is clearly multi-valued. Can we construct a riemann surface for $w=\log z$?
    \begin{tikzcd}  
    X \arrow[d,"\text{covering}"] \arrow[r, "\text{single value}"] & \C \\
    \C \arrow[ru,"\log z"] &
    \end{tikzcd}
    Let $X=\{(z,w)\in\C^2|z=e^w\}$ then again the single-valued function $(w,z)\to w$ is singled valued.
\end{itemize}
Now we can try to map the open strip $-i\pi/2<\Im(z)<i\pi/2$ to the unit disk. First we use $\zeta=e^z$ to map to the right half plane, then $w=\frac{\zeta-1}{\zeta+1}.$
\section{Analytic functions}
\begin{definition}[Analytic Function]
    A function $f$ is \bluebf{analytic} in an open set $\Omega$ if it has a convergent power series representation at every point $z_0\in\Omega.$

    i.e. $\forall\,z_0\in\Omega$ there is a power series $\sum a_n(z-z_0)^n$ such that $f(z)=\sum a_n(z-z_0)^n$ when $|z-z_0|<R$ for some $R>0.$
\end{definition}
\begin{itemize}
    \item If $f(z)$ has convergent power series representation at $z_0,$ then there is a convergent power series $g(z)$ at $z_0$ such that $g'(z)=f(z)$ in some disk $|z-z_0|<R,$ where $R$ is the radius of convergence of $f.$ We know \begin{align}
        f(z)&=\sum_{n=0}^\infty a_n(z-z_0)^n\\
        g(z)&=c+\sum_{n=0}^\infty\frac{a_n}{n+1}(z-z_0)^{n+1}\\
    \end{align}
    The primitive is uniquely determined up to a constant.
    \item \textbf{Question:} Does a convergent power series define an analytic function?
    \begin{proposition}
        If $f(z)=\sum a_nz^n$ is a convergent power series with radius of convergence $R,$ then $f(z)$ is analytic in $|z|<R.$
        \begin{proof}
            Note what we need to show. For any $z_0$ with $|z_0|<R,$ then $f(z)$ has convergent power series representation at $z_0$ with radius of convergence $R-|z_0|.$\begin{align}
                f(z)&=\sum a_nz^n\\
                &=\sum a_n(z_0+(z-z_0))^n\\
                &=\sum_{n=0}^\infty a_n\sum_{k=0}^n\binom{n}{k}z_0^{n-k}(z-z_0)^k
            \end{align}
            Note that if we take \begin{align}
                \sum_{n=0}^\infty|a_n|\left(|z_0|+|z-z_0|\right)^n&=\sum_{n=0}^\infty|a_n|\sum_{k=0}^n\binom{n}{k}|z_0|^{n-k}|z-z_0|^k
            \end{align}
            We know this series is absolutely convergent. So we can change the order of summation  to conclude \begin{align}
                f(z)=\sum_{k=0}^\infty\left(\sum_{n=k}^\infty a_n\binom{n}{k}z_0^{n-k}\right)(z-z_0)^k
            \end{align}
        \end{proof}
    \end{proposition}
    \item We notice that the inner sum\begin{align}
        \frac{1}{k!}f^{(k)}(z_0)=\sum_{n=k}^\infty a_n\binom{n}{k}z_0^{n-k}
    \end{align} is the $k$th derivative of $f$ at $z_0,$ so $f$ is holomorphic.
\end{itemize}
\subsection{Analytic Continuation}
\begin{theorem}
    Given $f(z)$ is analytic in a domain $\Omega$ and $z_0\in\Omega,$ then the following are equivalent\begin{enumerate}
        \item $f^{(n)}(z_0)=0$ for all $n\geq 0.$
        \item $f$ is identically $0$ in a neighborhood of $z_0.$
        \item $f$ is identically $0$ in $\Omega.$
    \end{enumerate}
    \begin{proof}
        $(3)\implies(1)$ is trivial, $(1)\implies(2)$ can be shown from the convergent power series representation of $f$ at $z_0$ (as the coefficient are the derivatives).
        
        To show $(2)\implies(3)$, we define \begin{align}
            \Omega'=\{z\in\Omega|f=0\text{ in a neighborhood of }z\text{ in }\Omega\}.
        \end{align}
        Clearly $\Omega'\neq\emptyset$ because $z_0\in\Omega.$

        $\Omega'$ is open by definition.

        $\Omega'$ is also closed. Take $z\in\overline{\Omega'}.$ Then, $f^{(n)}(z)=0$ for all $n\geq 0$ by continuity. Then $f=0$ in a neighborhood of $z,$ by $(1)\implies(2).$ So $z\in\Omega'.$ thus $\Omega'$ is closed. 

        Hence, $\Omega=\Omega'.$
    \end{proof}
    \begin{corollary}
        \begin{enumerate}
            \item If $f,g$ are analytic in domain $\Omega$ and $f=g$ in a neighborhood of some point then $f=g$ in $\Omega.$
            \item The ring $\mathcal A(\Omega)$ if analytic functions in a domain $\Omega$ is an \bluebf{integral domain}.
        \end{enumerate}
        \begin{proof}
            The proof of $(1)$ is trivial using $h=f-g.$ For $(2),$ suppose $f,g\in\mathcal{A}(\Omega)$ and $fg=0.$ Suppose $f\neq0$ then there is $z_0$ s.t. $f$ is non-vanishing in a neighborhood in a neighborhood $U$ of $z_0.$ So $g=0$ in $U$ hence $g=0$ in $\Omega.$
        \end{proof}
    \end{corollary}
\end{theorem}
\begin{itemize}
    \item Integral domains are good, but it is better to work with fields. Hence, we will now analyze the zeros and poles.
    \item Consider $f$ is analytic in a neighborhood of $z_0.$ Then $f(z)=\sum a_n(z-z_0)^n$ is a convergent power series with radius of convergence $R.$ Suppose $f(z_0)=0$ but $f\neq0.$
    \item Let $k$ be the smallest integer s.t. $f^{(k)}(z_0)\neq0$ (i.e. $a_k\neq0$) Then, we define $g$ s.t. $f(z)=(z-z_0)^kg(z).$ Then, \begin{align}
        g(z)=\sum_{n=k}^\infty a_n(z-z_0)^{n-k}
    \end{align}
    \item $k$ is the \bluebf{order} or \bluebf{multiplicity} of the zero at $z_0,$ characterized by $f^{(k)}(z_0)\neq0,$ but $f^{(j)}(z_0)=0$ for $j<k.$
    \item This shows the zero is \bluebf{isolated} meaning $f(z)\neq0$ in $0<|z-z_0|<\epsilon$ for any $\epsilon>0.$
    \item If we make a local change of variable near $z,$\begin{align}
        \zeta=(z-z_0)g(z)^{1/k}.
    \end{align}
    This is a change of coordinates because its derivative is nonzero. Then, $f(z(\zeta))=\zeta^k.$
    \item We now consider the quotients of analytic functions $f(z)/g(z)$ where $g$ is not identically zero. $f(z)/g(z)$ is well-defined and analytic in a neighborhood of $z_0$ if and only if $g(z)$ is analytic in a neighborhood of $z_0$ where $g(z_0)\neq0.$
    \item What if $g(z_0)=0$? We can try to factor out terms of $z-z_0$ so that $f_1(z_0)\neq0$ and $g_1(z_0)\neq0$ and\begin{align}
        f(z)&=(z-z_0)^k f_1(z)\\
        g(z)&=(z-z_0)^l g_1(z)
    \end{align}
    Then \begin{align}
        \frac{f(z)}{g(z)}=(z-z_0)^{k-l}\frac{f_1(z)}{g_1(z)}
    \end{align}
    We know that $f_1(z)/g_1(z)$ is analytic and nowhere vanishing in a neighborhood of $z_0.$ There are two cases \begin{itemize}
        \item $k\geq l$ then $f/g$ extends to be analytic at $z_0.$
        \item $k<l$ then $z_0$ is a \bluebf{pole} of $f/g$ of \bluebf{order} $l-k.$ Then, \begin{align}
            \left|\frac{f(z)}{g(z)}\right|\to\infty\text{ as }z\to z_0,
        \end{align}
        so $f/g$ still make sense as a function with values in the Riemann sphere.
    \end{itemize} 
\end{itemize}
\begin{definition}[Meromorphic Function]
    In an open set $\Omega,$ a \bluebf{meromorphic function} is well-defined and analytic in $\Omega\setminus D$ where $D$ is a discrete set, and expressible at in a neighborhood of any point of $\Omega$ as the quotient $f/g$ with $g$ is not identically zero. 
\end{definition}
\begin{itemize}
    \item Meromorphic functions in domain $\Omega$ form a \textbf{field}.
    \item \textbf{Exercise:} If $f(z)$ is meromorphic in $\Omega,$ then $f'(z)$ is also meromorphic in $\Omega$ with the same poles as $f.$ If $z_0$ is a pole of order $k$ of $f(z)$ then $z_0$ is a pole of order $k+1$ of $f'(z)$.
\end{itemize}


\end{document}
