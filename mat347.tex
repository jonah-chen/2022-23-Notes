\documentclass[a4paper]{article}
\usepackage{stdtemplate}

\title{MAT347 Abstract Algebra}
\author{Jonah Chen}
\date{}
\begin{document}
\maketitle
\sffamily
\section{Groups}
Groups are generally associated with symmetries. Consider the equilateral triangle:
\begin{center}
    \begin{tikzpicture}
        % draw a triangle and label the vertices A, B, C
        \draw (0,0) -- (4,0) -- (2,3.464) -- cycle;
        \node[above] at (2,3.464) {$A$};
        \node[below] at (4,0) {$B$};
        \node[below] at (0,0) {$C$};
    \end{tikzpicture}
\end{center}
We know that there are six symmetries of the triangle:
\begin{itemize}
    \item Identity transformation (do nothing) denoted as $\id$ or $e$
    \item Two rotations ($A\to B\to C\to A$ and $A\to C\to B\to A$)
    \item Three reflections % A <-> B
    $A \leftrightarrow B$, $A \leftrightarrow C$, $B \leftrightarrow C$
\end{itemize}
Note that these symmetries preserve the structure of the triangle, hence the composition of two symmetries must also be a symmetry. Let
\begin{itemize}
    \item $\rho$ be the rotation $A\to B\to C\to A$
    \item $\sigma$ be the reflections $B\leftrightarrow C$
\end{itemize}
Note that $\rho\sigma$ is the $A\leftrightarrow C$ reflection and $\sigma\rho$ is the $A\leftrightarrow B$ reflection. Hence they may not be commutative.

% Note that a square has eight symmetries. Then, that tells us that the square and triangle are two different shapes (this applies in general).

We also know that all symmetries can be reversed. $\alpha$ has an inverse $\alpha\inv$ such that $\alpha\alpha\inv = \alpha\inv\alpha = e$. These inspires the following definition:

\begin{definition}
    A \textbf{group} is a set $G$ with a composition
    \begin{align}
        G\times G &\to G \\
        (g, h) &\mapsto g\cdot h
    \end{align}
    Satisfying:
    \begin{itemize}
        \item Associativity: $(g\cdot h)\cdot k = g\cdot (h\cdot k)$
        \item Identity: $\exists\, e\in G$ such that $g\cdot e = e\cdot g = g$ for all $g\in G$
        \item Inverse: $\forall\, g\in G$, $\exists\, g\inv\in G$ such that $g\cdot g\inv = g\inv\cdot g = e$
    \end{itemize}
\end{definition}

Examples:
\begin{itemize}
    \item $\Z$ with $+$ is a group. It is associative, $e = 0$ and $g\inv = -g$.
    \item $\Z/n\Z$ with addition modulo $n$.
    \item If $F$ is a field, it implicitly has two group structures:
    \begin{itemize}
        \item Additive group: $(F,+)$ is a group. It is associative, $e = 0$ and $g\inv = -g$.
        \item Multiplicative group: $(F\setminus\{0\},\times)$ is a group. It is associative, $e = 1$ and $g\inv = 1/g$.
    \end{itemize}
    \item $GL(n,F)$ -- ``general linear group'' contains all invertiable $n\times n$ matrices.
    \item $SL(n,F)$ -- ``special linear group'' contains all invertiable $n\times n$ matrices with determinant $1$.
    \item $SO(n,F)$ -- ``special orthogonal group'' $=\{A\in SL(n,F)|A^t=A\inv\}$.
\end{itemize}
\subsection{Cyclic Groups}
One of the simplest groups is $\Z/n\Z$ for $n\in\N$ with the operation addition modulo n. This is known as the ``cyclic group of order $n$'' or $C_n$. i.e. for $n=8$, $5+7=4\,(\mathrm{mod}\,8)$, which we denote $\bar{5}+\bar{7}=\bar{4}$.

We know the inverse $\bar k\inv=\overline{n-k}$ for nonzero $k$ or $\bar 0\inv=\bar 0.$

Another way to express the cyclic group is $\bar k\leftrightarrow e^{2\pi i k/n}$ with multiplication operation. Then, \begin{align}
    \overline{k+n}=e^{2\pi i (k+n)/n}=e^{2\pi i k/n}e^{2\pi i n/n}=e^{2\pi i k/n}=\bar k.
\end{align}

% draw a circle and 8 equally spaced points on it
\begin{center}
    \begin{tikzpicture}
        \draw (0,0) circle (2cm);
        \foreach \x in {0,1,...,7}
        {
            % draw dot
            \draw (\x*45:2cm) circle (1pt);
        }
        % label the first point
        \node[below] at (0*45:2cm) {$1\leftrightarrow\bar 0$};
        % draw xy axis
        \draw (-3cm,0) -- (3cm,0);
        \draw (0,-3cm) -- (0,3cm);
    \end{tikzpicture}
\end{center}

\begin{definition}[Order]
    The \textbf{order} of a group $G$ is its cardinality denoted $\mathrm{ord}(G)$ or $|G|$. It could be a finite or infinite ordinal. In particular, $|C_n| = n$.
\end{definition}

\subsection{Quaternion Group}
The quaternion group $\mathbb{H}=\{\pm1,\pm i,\pm j,\pm k\}$ is a group of order $8$ with the multiplication operation. It has 

\begin{definition}[Subgroup]
    A \textbf{subgroup} of a group $G$ is a subset $H\subseteq G$ such that $H$ is a group. 
\end{definition}

\begin{definition}[Coset]
    If $G$ is a group and $H\leq G$, consider sets of the form
    \begin{equation}
        Hg=\{hg|h\in H\}
    \end{equation}
    This is a \textbf{right coset} of $H$.
\end{definition}
\begin{theorem}[Partitioning with Cosets]
    Consider $Hg$ and $Hg'$ for $g.g'\in G$. There are two cases:
    \begin{itemize}
        \item They might be disjoint: $Hg\cap Hg'=\emptyset$.
        \item They might intersect. Suppose $hg=h'g'$ for some $h,h'\in H$
        \begin{align}
            h\inv hg&=h\inv h'g'\\
            g&=h\inv h'g'\in Hg'
        \end{align}
        Similarly, $g'\in Hg$. Consider an arbitrary element of $kg\in Hg$ with $k\in H$. Then, $kg=kh\inv h'g'\in Hg'$ i.e. $Hg\leq Hg'$. Similarly, $Hg'\leq Hg$. Thus, $Hg=Hg'$.
    \end{itemize}
    The right cosets of $H$ partition $G$. In particular,
    \begin{equation}
        G = \bigsqcup Hg_i
    \end{equation}
\end{theorem}

For fixed $g$, if $hg=h'g$ for $h,h'\in H$ then $hgg\inv=h'gg\inv$ so $h=h'$. So in $Hg$, every element can be matched with an element of $H$. So, $|Hg|=|H|$.

\begin{theorem}[Lagrange]
    If $|G|<\infty$ and $H\leq G$, then $|H|\Big\vert|G|$
\end{theorem}
\begin{definition}[Index]
    For $H\leq G$, the \textbf{index} of $H$ in $G$ is $[G:H]=|G|/|H|$.
\end{definition}
If $|G|=13$, the only subgroups or $G$ are $\{e\},G$.

If $G=\Z$ and $H=2\Z$ (even numbers). Then $H+0=H$ is one coset, and $H+1=$ the odd integers is another coset. So, $\Z=(2\Z)\sqcup(2\Z+1)$.

\underline{Same for Left Cosets}
Interaction of left and right cosets? 

Consider the triangle group with rotations $e,\rho,\rho^2$ and reflections $\sigma_A, \sigma_B, \sigma_C$ Consider the subgroup $H=\{e,\sigma_A\}$.
\begin{align}
    He&=\{e,\sigma_A\}\\
    H\rho&=\{\rho,\sigma_B\}\\
    H\rho^2&=\{\rho^2,\sigma_C\}\\
    eH=\{e,\sigma_A\}\\
    \rho H&=\{\rho, \sigma_C\}\\
    \rho^2 H&=\{\rho^2,\sigma_B\}
\end{align}
Note that the left and right cosets are different. They are the same if the group is commutative.

\begin{definition}[Action]
    An \textbf{action} of a group $G$ on a set $X$ is a map \begin{align}
        G&\times X\to X \\
        (g,x)&\mapsto gx
    \end{align}
    such that 
    \begin{align}
        (gh)x&=g(hx)\\
        ex&=x
    \end{align}
\end{definition}

If $G$ is a group, it acts on itself. This is called a ``left translation'' or ``left regular action''.

How about the right action $(g,x)\mapsto xg$. The second condition may not be true\begin{align}
    (gh,x)=xgh\\
    (g,(hx))=(g,xh)=xhg
\end{align} which is not true. Instead, let $(g,x)=xg\inv$. Then,
\begin{align}
    (gh,x)=x(gh)\inv=xh\inv g\inv\\
    (g,(h,x))=(g,xh\inv)=xh\inv g\inv
\end{align}
This is the definition of the right action.

There is a third action of $G$ on itself by $(g,x)=gxg\inv.$ This action is called conjugation.

Take the following example: Let $G=SO(3)$ and let $X=S^2$. $G$ acts on $X$ by rotation. Let $H=\left\{\begin{pmatrix}
    \cos\theta & -\sin\theta & 0\\
    \sin\theta & \cos\theta & 0\\
    0 & 0 & 1
\end{pmatrix}\right\}$ be the subgroup of rotations that fixes the $z$-axis. 

$H$ also acts on $X$ ??
\begin{definition}[Orbit]
    If $G$ acts on $X$, the \textbf{orbit} of $x\in X$ is the set $Gx=\{gx|g\in G\}$. i.e. the set of all points $x$ is taken to by elements of $G$.
\end{definition}

The orbits of $H\approx SO(2)$ on the sphere are the lines of latitude (and the north and south poles). 

$H$ fixes the north pole, thus every coset $gH$ takes the north pole to a point. Suppose $gH$ and $g'H$ are cosets such that $gHN=g'HN\implies gN=g'N\implies (g')\inv gN=N\implies (g')\inv g\in H\implies gH...$ so the points ofn the sphere are in 1-1 correspondence with the left cosets of $H$. 
\begin{definition}[Stabilizer]
    If $G$ acts on $X$ and $x\in X$, the ``stabilizer'' of $x$ in $G$ is $\{g\in G|gx=x\}$
\end{definition}

\begin{definition}[Centralizer]
    If $A\subset G$, the \textbf{centralizer} of $A$ in $G$ is $C_G(A)=\{g\in G|ga= ag\forall a\in A\}$
\end{definition}
\begin{itemize}
    \item If $G$ is abelian, then $C_G(A)=G$ for any $A$. 
    \item In the triangle group, $C_G(\{\rho\})=\{e,\rho,\rho^2\}$
\end{itemize}
\begin{definition}[Center]
    The \textbf{center} of $G$ is $Z(G)=\{g\in G|gg'=g'g\forall g'\in G\}=C_G(G)$ 
\end{definition}
\begin{proposition}
    For any $A\subset G$, $C_G(A)\leq Z(G)$ (is a subgroup).
\end{proposition}

Consider the regular $n$-gon ($n\geq3$), what are its rigid motion symmetries?
\begin{itemize}
    \item There are always $n$ rotations by $\frac{2\pi}{n}$ about the origin.
    \item When $n$ is even, there are $n/2$ reflections in each pair of edges, and each pair of vertices. When $n$ is odd, there are $n$ reflections in each pair of (edge, vertex). There are always $n$ reflections.
    \item Write $\rho$ for clockwise rotation by $\frac{2\pi}{n}$. Fix one vertex and let $\sigma$ be the reflection that fixes that vertex.
    \item Note that $\rho\sigma=\sigma\rho\inv$. To show this, it suffices to find where two of the vertices gets mapped.
\end{itemize}
\begin{proposition}
    The symmetries are $e,\rho,\rho^2,\dots,\rho^{n-1},\sigma,\sigma\rho,\sigma\rho^2,\dots,\sigma\rho^{n-1}$
\end{proposition}
\begin{definition}[dihedral group]
    The group of symmetries of the regular $n$-gon is $D_{2n}$, the \textbf{dihedral group} of order $2n$.
\end{definition}

Given $H\leq G$ we write $G/H$ as the set of left cosets
\begin{align}
    G/H&=\{gH|g\in G\}\\
    H\setminus G&=\{Hg|g\in G\}
\end{align}
Both of these are called ``$G$ mod $H$''. In general, the two are \underline{different}. 

Now we want to ask, is $H\setminus G$ a group?
\begin{itemize}
    \item The most naive idea is to reuse multiplication in $G$, i.e. $Hg\cdot Hg'=Hgg'$, but it only sometimes works.
    \item This formula means: $hg\cdot h'g'=h''gg'$. For any $h,h'\in H,\exists h''$ s.t. this holds.
    \item Trick: $hg\cdot h'g'=hgh'eg'=hgh'(g\inv g)g'=h(ghg\inv)gg'$. Now we can ask if $ghg\inv\in H$ (for every $h'\in H$)
\end{itemize}
\begin{definition}[Normal Subgroup]
    A subgroup $H\leq G$ is \textbf{normal} if $ghg\inv\in H\forall g\in G,h\in H$, which is abbreviated as $gHg\inv=H$. $H\trianglelefteq G$ means $H$ is a normal subgroup of $G$
\end{definition}
\begin{itemize}
    \item Notice that if $gHg\inv=H$ then $gH=Hg$. So $H$ is normal, the left and right cosets must be the same.
\end{itemize}

\begin{definition}[Quotient Group]
    If $H\trianglelefteq G$, then $G/H$ is called the quotient group.
\end{definition}
\subsection{Homomorphisms}
\begin{definition}[Homomorphism]
    If $G,K$ are groups, a \textbf{homomorphism} is a map $\varphi:G\to K$ such that $\varphi(gg')=\varphi(g)\varphi(g')\,\forall g,g'\in G$.
\end{definition}
Observations: IF $\varphi:G\to K$ is a homomorphism and $g\in G$, then
\begin{enumerate}
    \item $\varphi(g)=\varphi(eg)=\varphi(e)\varphi(g)$, so $\varphi(e)=e$ (the identity element of $K$)
    \item $e=\varphi(e)=\varphi(gg\inv)=\varphi(g)\varphi(g\inv)$, so $\varphi(g\inv)=\varphi(g)\inv$
\end{enumerate}
Examples
\begin{itemize}
    \item $G=\Z$ and $\varphi:\Z\to\Z, \varphi(n)=2n$ is a homomorphism, as $\varphi(n+m)=2(n+m)=2n+2m=\varphi(n)+\varphi(m)$
    \item $G=\Z,K=\R$ and $\varphi:\Z\to\R,\varphi(n)=n$. This mapping is called an \bluebf{inclusion} as $Z\subset\R$.
    \item If $G$ is a group and $g_0\in G$, then $C_{g_0}:G\to G,g\mapsto g_0gg_0\inv$ is a homomorphism.
    \item A linear transformation $T:V\to W$ if $V,W$ are vector spaces (the additive group). 
    \item Note that $\varphi:g\mapsto g\inv$ is \bluebf{only} a homomorphism if $G$ is abelian.
\end{itemize}
\begin{definition}[Kernel/Image]
    If $\varphi:G\to G'$ is a homomorphism, then the \textbf{kernel} of $\varphi$ is\begin{equation}
        \ker(\varphi)=\{g\in G|\varphi(g)=e\}.
    \end{equation}
    The \textbf{image} of $\varphi$ is\begin{equation}
        \mathrm{im}(\varphi)=\{\varphi(g)|g\in G\}\subseteq G'
    \end{equation}
\end{definition}
\begin{theorem}
    $\ker(\varphi)\leq G$ and $\mathrm{im}(\varphi)\leq G'$ $\ker(\varphi)\trianglelefteq G$
    \begin{proof}
        Since $\varphi(e)=e$, $e\in\ker(\varphi)$, and $e\in\mathrm{im}(\varphi)$.
        So both are nonempty.

        Suppose $g,h\in\ker(\varphi)$, $e=\varphi(e)=\varphi(hh\inv)=\varphi(h)\varphi(h\inv)$ \dots
    \end{proof}
\end{theorem}
\begin{itemize}
    \item Suppose $N\trianglelefteq G$ and then define $G\to G/N, g\mapsto Ng$. We claim this is a homomorphism. Proof is simple $\varphi(gg')=Ngg'$, $\varphi(g)\varphi(g')=NgNg'=NgN(g\inv gg')=N(gNg\inv)gg'=NNgg'=Ngg'$ 
    \item This map is called the (natural) \bluebf{projection} of $G$ onto $G/N$. Sometimes written $\Pi_{G/N}$ or $\mathrm{proj}_{G/N}$.
    \item $\mathrm{im}(\Pi_{G/N})=G/N$ and $\ker(\Pi_{G/N})=N$.
    \item Any homomorphism is related to this one, so this could be considered as the ``generic homomorphism''.
\end{itemize}
\begin{definition}[Isomorphism]
    If $\varphi:G\to H$ is a homomorphism, and $\ker(\varphi)=\{e\}$ then $\varphi$ is injective.
    If $\varphi(G)=H$ then $\varphi$ is surjective. Thinking of $G$ and $H$ as sets, there is an inverse $\varphi\inv:H\to G$ such that $\varphi\inv\circ\varphi=1_G$ and $\varphi\circ\varphi\inv=1_H.$ It is easy to check that $\varphi\inv$ is also a homomorphism.

    In this case, $\varphi$ is an \textbf{isomorphism}
\end{definition}
\begin{itemize}
    \item Suppose we have an injective homomorphism $\varphi:G\to H$ where $\ker(\varphi)=\{e\}.$ Then, we can consider $\varphi:G\to\mathrm{im}(\varphi)<H.$ Sometimes we say $\varphi:G\to H$ is an \bluebf{isomorphism into} $H,$ as opposed to an isomorphism \bluebf{onto} $H$ or between $G$ and $H.$
\end{itemize}
\begin{definition}[Automorphism]
    If $G$ is a group, an \textbf{automorphism} of $G$ is an isomorphism $\varphi:G\to G.$
\end{definition}
Examples:
\begin{itemize}
    \item If $G=\Z,n\mapsto-n$ is the only automorphism apart from the identity.
    \item If $G$ is abelian, $g\mapsto g\inv$ is an automorphism.
    \item If $F$ is a field, and $G=GL(n,F)$ then $g\mapsto(g^t)\inv$ (transposed inverse) is an automorphism.
    \item If we fix $g_0\in G$ then the conjugation $C_{g_0}:G\to G$ where $C_{g_0}(g)=g_0g{g_0}\inv$ is an automorphism. 
\end{itemize}
\begin{definition}[Automorphism Group]
    $\mathrm{Alt}(G)$ is the \textbf{group} of automorphisms of $G$.
\end{definition}
\begin{definition}[Inner/Outer Automorphisms]
    The \textbf{inner automorphisms} of $G$ are \begin{equation}
        \textrm{Inn}(G)=\{\varphi\in\mathrm{Alt}(G)|\varphi=C_{g_0}\text{ for some }g_0\in G\}.
    \end{equation}
    If an element of $\mathrm{Alt}(G)$ that is not inner is \textbf{outer}.
\end{definition}
\begin{itemize}
    \item It is easy to show that $\mathrm{Inn}(G)\leq\mathrm{Alt}(G).$ 
    \item Observe that if $G$ is abelian, then $\mathrm{Inn}(G)=\{\mathrm{id}\}$
    \item In general, $\{\mathrm{id}\}\leq\mathrm{Inn}(G)\leq\mathrm{Alt}(G).$
\end{itemize}
\begin{itemize}
    \item The map \begin{align}
        G&\to\mathrm{Alt}(G)\\
        g&\to C_g
    \end{align}
    is a homomorphism. Its image is $\mathrm{Inn}(G)$ and its kernel is $Z_G$ (the center).
\end{itemize}
\begin{definition}[Fiber]
    If $p$ is a projection, then $p\inv(x)$ is the \textbf{fiber} over $x$
\end{definition}
\begin{itemize}
    \item If $N\triangleleft G,$ the projection $\pi:G\to G/N$ is a homomorphism. The fibers of $\pi$ is the cosets $gN=Ng,$ and they are all the same size.
    \item Suppose $\varphi:G\to H$ is a homomorphism, and $N=\ker(\varphi)\trianglelefteq G.$ The fibers of $\varphi$ is the cosets of $G/N.$
    \item We have $\varphi:G\to H$ and $\pi:G\to G/N.$ Wouldn't it be nice if $G/N\to H$ ``induced by $\varphi$'' were a homomorphism? Well, it is. 
\end{itemize}
\begin{theorem}[(First) Isomorphism]
    If $\varphi:G\to H$ is a homomorphism, and $N=\ker(\varphi),$ then there is a homomorphism $\bar\varphi:G/N\to H$ such that $\bar\varphi\circ\pi=\varphi.$ 
    
    Moreover, $\ker(\bar\varphi)=\{eN\},$ the trivial subgroup of $G/N,$ so $\bar\varphi$ is injective. So, $\bar\varphi:G/N\to\mathrm{im}(\varphi)$ is an \bluebf{isomorphism}.
\end{theorem}
\begin{itemize}
    \item This theorem suggests that you can construct an isomorphism from an arbitrary homomorphism. First, $\varphi$ factors through $G/N,$ then we can include it into $H.$
    \begin{equation}
        G\to^\pi G/N\to^{\bar\varphi}\mathrm{im}(\varphi)\to^{\text{inclusion}}H
    \end{equation}
\end{itemize}

\begin{theorem}[(Third) Isomorphism]
    $N\trianglelefteq G$ and $H\leq G,$ then $N\leq H\implies N\trianglelefteq G.$
    \begin{proof}
        ????
    \end{proof}
\end{theorem}
\begin{theorem}
    \begin{equation}
        G/H\cong G/N\Big/H/N
    \end{equation}
    \begin{proof}
        Define $\varphi:G\to G/N\Big/H/N$ by \begin{align}
            \varphi(g)=(gN)H/N
        \end{align}
        We need to show $\varphi$ is a homomorphism. Let \begin{align}
            \varphi(gg')&=gg'N\:H/N\\
            &=gNg'N\:H/N\\
            &=gN\:H/N\cdot g'N\:H/N\\
            &=\varphi(g)\varphi(g')\\
        \end{align}
    \end{proof}
    We will then ask what is $\ker(\varphi).$ Suppose $\varphi(g)=H/N,$ so $gN\:H/N=H/N.$ But $g$ is a representation for $gN,$ so $gH/N$ for this to be in $H/N$ we want $g\in H$ so $\ker(\varphi)=H.$ An arbitrary element of $G/N\Big/H/N$ is $gN\:H/N$ for some $g\in G,$ so $\mathrm{im}(\varphi)=G/N\Big/H/N.$
\end{theorem}
\begin{itemize}
    \item $G=\Z,H=3\Z,K=4\Z.$ By the second isomorphism theorem, $\Z/3\Z\cong 4\Z/12\Z,$ and also $Z/4\Z\cong 3\Z/12\Z.$
\end{itemize}
\begin{definition}[Equivilence Class]
    Being in the same coset of a subgroup $H$ is an equivalence relation. So, the large group is a disjoint union of equivalence classes (cosets) of $H.$
\end{definition}
\begin{itemize}
    \item The cosets of $\Z$ in $\R$ is $r+\Z$ for $r\in[0,1).$
    \item Homomorphism $\varphi:\R\to\C^\times, t\mapsto e^{2\pi it}.$ Then, $\ker(\varphi)=\Z.$ Observe tat $\varphi$ is \bluebf{onto} the unit circle, by the first isomorphism theorem, $\R/\ker(\varphi)=\R/\Z\cong S^1.$
    \item $\Z^2\triangleleft\R^2$
\end{itemize}
\begin{theorem}[Fourth Isomorphism Theorem/Lattice Theorem]
    Consider a lattice of subgroups with $N\trianglelefteq G.$ In $G/N,$ the subgroup lattice has the same structure as the subgroup lattice of $G$ that contains $N.$

    Specifically, if $N\trianglelefteq G,$ and $N\trianglelefteq H<G,$ we write $\bar H=H/N.$ Including $\bar G=G/N$ and $\bar N=\bar e=N/N.$

    Then, the lattice of $\bar H$s in $\bar G$ has the same lattice structures as the part of the lattice for $G$ consisting of subgroups that are intermediate between $N$ and $G.$ Moreover,\begin{align}
        H\leq K&\iff\bar H\leq\bar K\\
        H\trianglelefteq K&\iff\bar H\trianglelefteq\bar K\\
        [H:K]&=[\bar H:\bar K] \text{ if }K\leq H\\
        \overline{H\cap K}&=\bar H\cap\bar K\\
        \overline{\langle H,K\rangle}&=\langle\bar H,\bar K\rangle
    \end{align}
\end{theorem}
If $G,G'$ are groups, consider the cartesian product $G\times G'=\{(g,g')|g\in G,g'\in G'\}.$ Note that $|G\times G'|=|G||H|.$ There is an obvious way to turn this into a group by\begin{align}
    (g,g')(h,h')&=(gh,g'h')\\
    (g,g')^{-1}&=(g^{-1},g'^{-1})
    e=(e,e)
\end{align}
In $G\times G',$ the subset $G_0=:\{(g,e)|g\in G\}\cong G$ is a subgroup. Likewise, $G'_0=:\{(e,g')|g'\in G'\}\cong G'.$ Also notice that $G_0$ and $G'_0$ commute. So, $(G\times G')/G_0\cong G'.$

\subsection{Symmetric Groups}
\begin{definition}[Symmetric Group]
    The symmetric group $S_n$ is the group of permutation of $n$ elements, with composition as the operation.
\end{definition}
\begin{itemize}
    \item $|S_n|=n!$
    \item A cycle is a permutation that cycles through some subset of $\{1,\dots,n\}$, denoted as \begin{align}
        (a_1\,a_2\,\dots\,a_k),\quad k\leq n\text{ and }a_i\text{ are distinct.}
    \end{align}
    Represents the permutation $a_1\to a_2\to\dots\to a_k\to a_1.$
    \item Note that these are ambiguous, as $(a_1\,a_2\,\dots\,a_k)$ is the same as $(a_2\,a_3\,\dots\,a_k\,a_1).$ So by convention, we often start with the smallest number first so they are unique. 
    \item $k$ is the length of the cycle, it is also called a \bluebf{$k$-cycle}.
    \item Every permutation can be written as a product of disjoint cycles. If given a permutation, we will start from $1$ and write a cycle until we get back to $1.$ Then, we will start from the next number that hasn't been included yet and repeat until we get to the end.
    \item If $\sigma=(1\,3\,6)(4\,5),$ then $\sigma\inv=(4\,5)\inv(1\,3\,6)\inv=(4\,5)(1\,6\,3)=(1\,6\,3)(4\,5).$ We will order the cycles by their first element, and omit $1$-cycles.
    \item Two \bluebf{disjoint cycles} (i.e. without any numbers in common) will commute.
    \item If cycles are not disjoint, like $\sigma=(1\,4\,2)(2\,3\,5)(3\,4\,7)\in S_7$ will not commute.
    \begin{itemize}
        \item $1\to4$
        \item $4\to7$
        \item $7\to3\to5$
        \item $5\to2\to1$
        \item $2\to3$
        \item $3\to4\to2$
    \end{itemize}
    So $\sigma=(1\,4\,7\,5)(2\,3).$
    \item Any $k$-cycle is a product of $2$-cycles. Thus, every element in the symmetric group can be written as a product of $2$-cycles so $S_n$ is generated by $2$-cycles. For example, if $k=4$ and $\sigma=(a\,b\,c\,d),$ then $\sigma=(a\,d)(a\,c)(a\,b).$
    \item We can ask what is the minimum number of $2$-cycles needed to generate any $\sigma\in S_n.$ In general, this is a very difficult question to answer. However, the \bluebf{parity} of the number of $2$-cycles in a product equalling $\sigma$ is well-defined.
    \item If $\sigma=(a_1\,b_1)(a_2\,b_2)\dots(a_k\,b_k)$ is a product of $2$-cycles, then $\sigma$ is \bluebf{even} if $k$ is even, and \bluebf{odd} if $k$ is odd.
    \item \redbf{Warning: a $k$-cycle is even if $k$ is odd, and odd if $k$ is even.}
    \item To make odd and even well defined, we need to know that the parity of a permutation is independent of the way we write the cycles.
    \begin{proof}
        Given $\sigma\in S_n$ is a $k$-cycle. Define $\Delta=\prod_{1\leq i<j\leq n}(j-i).$ If $\tau\in S_n$, it acts on $\Delta$ with \begin{align}
            \tau\cdot\Delta=\prod_{1\leq i<j\leq n}(\tau(j)-\tau(i)).
        \end{align}
        These two products are the same up to a factor of $\pm1,$ you have to multiply by $-1$ for each pair $i<j$ for which $\tau(i)>\tau(j).$
        
        We will consider how $(a\,b)$ with $a<b$ affect $\Delta.$ If neither $i$ nor $j$ is equal to $a$ or $b,$ the term is unaffected. Note that \begin{itemize}
            \item If $i<a,$ then $i<\tau(a)=b$ and $i<\tau(b)=a.$ So $(i\,a)$ or $(i\,b)$ are unaffected.
            \item Likewise, for $j>b$ then $(a\,j)$ or $(b\,j)$ are unaffected.
        \end{itemize}
        The only pairs that will be affected are ones $(a\,i),(i\,b)$ with $a<i<b$ and $(a\,b).$ If $a<i<b,$ then both $(a\,i)$ and $(i\,b)$ will change sign, so the product will be unaffected. $(a\,b)$ will change sign, so $\Delta$ will change sign under a transposition.

        If $\sigma\in S_n,$ write it as any product of $k$ transpositions. If $\sigma\cdot\Delta=\Delta$ then there must be an even number of transpositions. If $\sigma\cdot\Delta=-\Delta$ then there must be an odd number of transpositions. Thus, the parity of $\sigma$ is independent of the way we write it.
    \end{proof}
\end{itemize}
\begin{definition}[Sign]
    The sign of $\sigma\in S_n$ is \begin{equation}
        \sgn(\sigma)=(-1)^k,
    \end{equation}
    if $\sigma$ is a product of $k$ transpositions. 
\end{definition}
\begin{itemize}
    \item Note that $\sgn(\sigma\tau)=(-1)^k(-1)^l=(-1)^{k+l}=\sgn(\sigma)\sgn(\tau).$
    \item Thus, $\sgn:S_n\to\{\pm1\}$ is a homomorphism.
    \item $\ker(\sgn)=A_n\trianglelefteq S_n$ is the alternating group of $k$ elements which contains all the even permutations. Note that
    \begin{align}
        S_n/A_n\cong\{\pm1\}\quad[S_n:A_n]=2\quad|A_n|=\frac{n!}{2}
    \end{align}
    \item for $n>5,$ $A_n$ has no normal subgroups. What are the possible cycle types in $A_5$? There is $(a\,b\,c\,d\,e),(a\,b)(c\,d),(a\,b\,c)$
    \item Let $\sigma\in S_n$ with $a\to b\to c\to\cdots,$ and suppose $\tau\in S_n$ takes $a\to a',b\to b',c\to c',\dots.$ Consider the conjugation $\tau\sigma\tau\inv.$\begin{align}
        \tau\sigma\tau\inv(a')=\tau\sigma(a)=\tau(b)=b'\\
        \tau\sigma\tau\inv(b')=\tau\sigma(b)=\tau(c)=c'\\
    \end{align}
    So $\tau\sigma\tau\inv$ takes $a'\to b'\to c'\to\cdots.$ Conjugating by $\tau$ ``relabels'' what $\sigma$ by replacing $a$ with $a',$ \dots.
\end{itemize}
\subsection{Simple Group}
One way we study groups is to write it as a chain of normal subgroups $G_0=\{e\}\triangleleft G_1\triangleleft G_2\triangleleft\cdots\triangleleft G_r=G,$ where $G_{i+1}/G$ is a simple group $\forall i=0,\dots,r-1.$ A decomposition like this is called a \bluebf{Jordan-Holder Series} (composition series), and the quotients are called the \bluebf{composition factors}. However, the same $G$ may have different composition series. \begin{theorem}[Jordan-Holder]
    Any two Jordan-Holder series for $G$ have the same length. Moreover, the composition factors are the same (but perhaps in different orders).
\end{theorem}

Example: Suppose $H,K$ are both normal subgroups of $G.$ Apply 2nd isomorphism theorem. Note, $H\subseteq Norm_G(K)=G$ and $K\subseteq Norm_G(H)=G.$ Thus, $HK/K\cong H/H\cap K$ and $HK/H\cong K/H\cap K.$ In this example there are two composition series\begin{align}
    \{e\}\triangleleft H\cap K\triangleleft H\triangleleft HK\triangleleft G\\
    \{e\}\triangleleft H\cap K\triangleleft K\triangleleft HK\triangleleft G
\end{align}

Every group has a Jordan-Holder series. In genera, a group $G$ is not determined by its Jordan-Holder series. However, if $G$ is simple, then its Jordan-Holder series is $\{e\}\triangleleft G.$

\begin{definition}[Solvable]
    If the composition factor $G_{i+1}/G_i$ of $G$ are all \textbf{abelian}, we say $G$ is \bluebf{solvable}.
\end{definition}

If $G$ acts on a set $X,$ then each $g\in G$ permutes the element of $X.$ So there is a map $G\to S_X$ (the symmetric group of $X$). It is easy to show that this map is a homomorphism. So, we will allow ourselves to go between group actions and Homomorphisms into $S_X.$

Suppose $H\leq G$ and let $X=G/H$ be the coset space. So, $G$ acts on $X$ by left multiplication $g(xH)\mapsto gxH.$ If $n=[G:H]=|X|,$ the action amounts to a homomorphism $\varphi:G\to S_n.$

Our first observation is that $G$ acts \textbf{transitively}. For any $x,y\in X,\,\exists\,g\in G$ s.t. $gx=y.$ i.e. the orbit of any $x\in X$ is $X.$

What is $\ker\varphi?$ We know that if $h\in\ker\varphi,$ that $hxH=xH.$ Then consider $h',h''\in H$ then \begin{align}
    hxh'=xh''\\
    hx=xh''{h'}\inv\\
    h=xh''{h'}\inv x\inv\\
    \ker\varphi=\bigcap_{x\in G}xHx\inv
\end{align}
If $H=\{e\},$ then $G/H=G,$ so $\ker\varphi=\{e\}.$ then $\varphi$ is injective. By the first isomorphism theorem, $G\cong \mathrm{im}\,\varphi=S_n.$
\begin{theorem}[Cayley]
    Any group $G$ with $|G|=n$ is isomorphic to a subgroup of $S_n.$
    \begin{proof}
        We already proved it!
    \end{proof}
\end{theorem}
Another example is to let $G$ act on itself by conjugation. In this case, $\varphi$ with $g\cdot x=gxg\inv=C_g(x).$ This is not a transitive action unless $G$ is trivial.  The orbits of conjugation are the \bluebf{conjugacy classes} of $G.$ They are disjoint (because conjugacy is an equivalence relation). 

Note that $geg\inv=e,\,\forall g.$ If $z\in Z(G),$ then $gzt\inv=zgg\inv=z\,\forall g,$ then the conjugacy classes contain a single element.

If $G$ is abelian, $Z(G)=G$ and every element is its own conjugacy class.

Because conjugacy is an equivalence relation, $G$ is a disjoint union of all conjugacy classes.

If $Z(G)=\{e,z_1,\dots,z_k\}$ and $g_1,\dots,g_m$ are representatives from the non-central conjugacy classes. Let's write $C(g_i)=\{gg_ig\inv|g\in G\}.$ So,\begin{align}
    G=Z(G)\sqcup\left(\bigsqcup C(g_i)\right)
\end{align}
so \begin{align}
    |G|=|Z(G)|+\sum_i|C(g_i)|
\end{align}
This is called the \bluebf{Class Equation}.\begin{theorem}[Orbit-Stabilizer]
    If $G$ acts on $X,$ for each $x\in X,$ write $G\cdot x$ for its orbit. Then,\begin{align}
        |G\cdot x|=[G:G_x]=[G:\mathrm{Stab}(x)]
    \end{align}
    The point is that two things in the same coset of $G_x$ has the same effect on $x$.
\end{theorem}
Under conjugation, \begin{align}
    \mathrm{Stab}(x)=G_x=\{g\in G|gxg\inv=x\}=Z(x),
\end{align}
the centralizer of $x$. So the class equation can be rewritten as \begin{align}
    |G|=|Z(G)|+\sum_{i}[G:Z(g_i)]
\end{align}
\begin{definition}[$p$-group]
    Suppose $p$ is prime, $G$ is a \bluebf{$p$-group} if $|G|=p^k$ for some $k\geq 1.$
\end{definition}
\begin{theorem}
    If $G$ is a non-trivial $p$-group, then it has a non-trivial center.
    \begin{proof}
        Suppose $|G|=1.$ Then \begin{align}
            |G|=|Z(G)|+\sum_{i}[G:Z(g_i)]
        \end{align}
        Claim $Z(g_i)<G$, otherwise $g_i\in Z(G).$ By Lagrange's theorem $|Z(g_i)|\mid|G|=p^k.$ So $|Z(g_i)|=p^l$ for some $l<k.$ Then,\begin{align}
            p^k=|G|=|Z|+\sum_{i}[G:Z(g_i)]\\
        \end{align} 
        Since $|Z|=1,$ the RHS is not divisible by $p$ so this is a contradiction.
    \end{proof}
    \begin{corollary}
        Suppose $p$ is prime. If $|G|=p^2,$ then $G$ is abelian.
        \begin{proof}[Proof]
            We know $Z(G)$ is a non-trivial subgroup so $1\neq|Z(G)|\Big|p^2.$ So $|Z(G)|=p$ or $p^2.$ If $|Z(G)|=p^2,$ then $G$ is abelian by definition. If $|Z(G)|=p,$ then $|G/Z(G)|=p$ hence $G/Z(G)\cong C_p.$ So $x\notin Z(G),$ then $G/Z(G)=\{\bar e, \bar x, \bar x^2,\dots\bar x^{p-1}\}$ where $\bar x=:xZ(G).$ Also, $\bar x^p=\bar e\in G/Z(G).$ Note that $\mathrm{ord}(x)$ is either $p$ or $p^2$.
            \begin{itemize}
                \item If $|\langle x\rangle|=p^2$ so $\langle x\rangle=G$ and $G$ is cyclic hence abelian. 
                \item If $\mathrm{ord}(p),$ then $G=\bigcup_{k=0}^{p-1}x^kZ(G).$ Recall $|Z(G)|=p,$ so $Z(G)$ is cyclic. Then, \begin{align}
                    Z(G)=\{e,z,z^2,\dots,z^{p-1}\}
                \end{align}
                so \begin{align}
                    G=\{x^iz^j|0\leq i,j<p\}
                \end{align}
                These elements commute. $x^iz^jx^mz^n=x^ix^mz^jz^n=x^{i+m}z^{j+n}$
            \end{itemize}
        \end{proof}
    \end{corollary}
\end{theorem}
Note we need to be careful with the steps in this proof. Just because $\bar x^p=\bar e$ doesn't mean there is a representative $x\in\bar x$ that is order $p.$

\begin{itemize}
    \item Now we consider the rotations of a tetrahedron. A easy way to think about this is to identify a ``top'' vertex, which is well defined ($4$ possibilities). Then, we fix the top and we have $3$ rotations (like of the triangle). So, there are $12$ rotations.
    \item Apart from $e$, there are two non-trivial rotations that fix any particular vertex. This only accounts for $8$ rotations, and $e$, so we are missing $3$ rotations. 
    \item The other rotations does not fix any vertices and are like $(1\,2)(3\,4).$ Then, $2,3,4$ goes with $1$ so we have $3$ rotations. This accounts for all 12.
    \item In summary, we have $e,$ and $8$ rotations in the form $(a\,b\,c)$ and $3$ rotations in the form $(a\,b)(c\,d).$ This is $A_4.$
    \item The rigid motions are $S_4.$
\end{itemize}
\begin{proposition}
    $A_5$ is simple. $A_5\triangleleft S_5$ with index $2,$ so $|A_5|=60.$
    
    \begin{proof}
        We will enumerate the conjugacy classes of $S_5$ \begin{itemize}
            \item $(a\,b\,c\,d\,e)\in A_5$
            \item $(a\,b\,c\,d)\notin A_5$
            \item $(a\,b\,c)\in A_5$
            \item $(a\,b\,c)(d\,e)\notin A_5$
            \item $(a\,b)(c\,d)\in A_5$
            \item $(a\,b)\notin A_5$
            \item $e\in A_5$
        \end{itemize}
        There are $24$ elements in the conjugacy class of $(a\,b\,c\,d\,e).$ However, $24$ does not divide $60$ so it is not a conjugacy class of $A_5.$

        Consider the centralizer $Z_{A_5}(a\,b\,c\,d\,e)\geq\left\langle(a\,b\,c\,d\,e)\right\rangle$ which has order $5$. But $Z_{A_5}(a\,b\,c\,d\,e)\leq Z_{S_5}(a\,b\,c\,d\,e)$ so $Z_{A_5}(a\,b\,c\,d\,e)=\left\langle(a\,b\,c\,d\,e)\right\rangle$

        So there are two $A_5$ conjugate classes of $5$-cycles, each with $12$ elements.

        There are $20$ $3$-cycles in $S_5.$ Are they all conjugate in $A_5?$ If $(a\,b\,c)$ is conjugate to $(x\,y\,z)$ by $\sigma\in S_5$, then it is also conjugate by $\sigma(d\,e)$ If $\sigma\notin A_5$ then $\sigma(d\,e)\in A_5$ so there is one conjugate class of $20$ $3$-cycles. 

        There are $15$ double transpositions.

        \textbf{If we have a normal subgroup, it is a union of the conjugacy classes,} so if $A_5$ has a normal subgroup it must be a combination of $1+15,20,12,12$ but there is no combination (apart from $1$) that divides $60.$ Hence, $A_5$ is simple. 
    \end{proof}
\end{proposition}
\section{Sylow Theorems}
\begin{theorem}
    Suppose $|G|=p^\alpha n$ where $p\not|n.$ Then, a subgroup $P\leq G$ is a Sylow $p$-subgroup if $|P|=p^\alpha.$ We'll write $n_p(G)$ for the number of Sylow $p$-subgroups of $G.$ \begin{enumerate}
        \item Sylow $p$-subgroups exist.
        \item Suppose $P$ is a Sylow $p$-subgroup of $G$ and $Q\leq G$ s.t. $|Q|=p^r$ for some $r>0.$ Then, $\exists\,g\in G$ s.t. $gQg\inv\subseteq P.$ In particular, all Sylow $p$-subgroups of $G$ are conjugate. 
        \item $n_p(G)\equiv1\mod{p}$ and $n_p(G)=[G:\mathrm{Norm}_G(P)]$ for any Sylow $p$-subgroup. Hence $n_p(G)||G|.$ It actually also divides $n=|G|/|P|.$
    \end{enumerate}
\end{theorem}
Before proving the theorem we will consider the following example: Let $G=S_3.$ The Sylow $2$ subgroups are $\{e,(1\,2)\},\{e,(1\,3)\},\{e,(2\,3)\}$ we know $n_2(S_3)=3\equiv1\mod2.$ The only Sylow $3$ subgroup is $A_3,$ so $n_3(S_3)=1.$\begin{lemma}
    If $G$ is abelian and $p\Big||G|,$ then $G$ contains an element of order $p.$
    \begin{proof}
        If $|G|=p$ then $G$ is cyclic and every non-trivial element has order $p$.

        If $|G|>p,$ and $x\in G$ with order $p^rm$ where $p\not{\big|}m.$ If $r\neq0,$then $x^{p^{r-1}m}$ has order $p.$
        
        This reduces us to the case where $p\nmid\mathrm{ord}(x), \forall x\in G.$ We will use induction. \begin{itemize}
            \item Assume the result is true for all groups smaller than $G.$ 
            \item If $p\nmid\mathrm{ord}(x)=|\langle x\rangle|<|G|.$ As $G$ is abelian, then $N=:\langle x\rangle\triangleleft G.$ 
            \item By induction $G/N$ contains an element of order $p.$
            \item i.e. $\exists y=y_0N\in G/N$ s.t. $y^p=e=N.$ so $y_0^p\in N$
            \item We claim that $\langle y_0^p\rangle<\langle y_0\rangle$ since otherwise $y_0\in N$ which has order $1.$
            \item This means $p\mid|y_0|$ otherwise $\langle y_0^p\rangle=\langle y_0\rangle.$ This is a contradiction.
            \item This means a suitable power of $y_0$ must have order $p.$
        \end{itemize}
    \end{proof}
\end{lemma}
\begin{lemma}
    If $P\in\mathrm{Syl}_p(G)$ and $Q$ is a non-trivial $p$-subgroup of $G.$ Then, $Q\cap\mathrm{Norm}_G(P)=Q\cap P.$
    \begin{proof}
        Let $H=Q\cap\mathrm{Norm}_G(P)\geq Q\cap P.$ We need to show that $H\leq Q\cap P.$
        But $H\leq Q$ so we only need to show that $H\leq P.$

        $H\leq N_G(P)\implies HP$ is a subgroup. The result will follow if we can argue that $HP$ is a $p$-group. We know that \begin{align}
            |HP|=\frac{|H||P|}{|H\cap P|}
        \end{align}
        Since $|H|,|P|,|H\cap P|$ are all powers of $P.$ So $HP\geq P$ but $|HP|$ can't be bigger than $|P|.$
    \end{proof}
\end{lemma}

Proof that Sylow $p$-subgroup exists. We will use induction on $|G|.$

If $p\mid |Z(G)|,$ we know by the lemma that $\exists z\in Z(G)$ with $|z|=p.$ Let $N=\langle z\rangle$ is a normal subgroup as $N\leq Z(G).$ Then, $G/N$ is a smaller group than $G.$ By the induction hypothesis say $G/N$ has a Sylow $p$-subgroup.

If $|G|=p^\alpha m,p\nmid m$ then $G/N=p^{\alpha-1}m.$ So, it has a Sylow $p$-subgroup of order $p^{\alpha-1}.$ By the lattice isomorphism theorem, the preimage of this group in $G$ has order $p^\alpha,$ as required.

Assume $p\nmid|Z(G)|.$ Let $g_1,\dots,g_k$ be representatives of the non-central conjugacy classes of $G.$ So,\begin{align}
    |Z(G)|+\sum_{i=1}^k[G:C_G(g_i)]=|G|.
\end{align}
We know that $p\mid |G|$ but $p\nmid |Z(G)|$ meaning for some $i,$ we know $p\nmid[G:C_G(g_i)].$ As $g_i$ represents a non-central conjugacy class, then $C_G(g_i)<G.$ We will use the induction hypothesis. Note that since $p\nmid[G:C_G(g_i)],$ then $p^\alpha\mid|C_G(g_i)|.$ By the induction hypothesis, $C_G(g_i)$ has a Sylow $p$-subgroup of order $p^\alpha,$ it is also a Sylow $p$-subgroup of $G.$ Thus, Sylow $p$-subgroups exist.

Fix a Sylow $p$-subgroup $P_1$ of $G$ and enumerate all its distinct conjugates as $P_1,\dots,P_r.$ Let $Q$ be any $p$-subgroup.

$G$ acts on $\mathcal S=\{P_1,\dots,P_r\},$ and $Q$ also act on $\mathcal S,$ but it may not have a single orbit. Decompose $\mathcal S$ into $Q$ orbits, \begin{align}
    S=\mathcal O_1\sqcup\mathcal O_2\sqcup\cdots\sqcup\mathcal O_s.
\end{align}
How big is $\mathcal O_k$? Relabel $P_1,\dots P_k$ s.t. $\mathcal O_k=\{qP_kq\inv|q\in Q\}.$ We know how to find the size of a conjugacy class. $|\mathcal O_k|=[Q:N_Q(P_k)].$ Note that $N_Q(P_k)=N_G(P_k)\cap Q.$ The second lemma states $N_G(P_k)\cap Q=P_k\cap Q.$

For now, let $Q=P_1.$ So, $\mathcal O_1=\{qP_1q\inv|q\in P_1\}=P_1.$ \begin{align}
    |S|=r=\underbrace{|\mathcal O_1|}_{1}+\underbrace{\sum_{i=2}^s[P_1:P_1\cap P_i]}_{\text{divisible by }p}
\end{align}
If we know that if all Sylow $p$-subgroup are conjugate, then we know the number of Sylow $p$-subgroup is $1\mod p.$

Let $Q$ be any $p$-subgroup of $G$ and suppose $Q$ is not contained in any of the $P_1,\dots P_r.$ Then, $Q\cap P_i$ is a proper subgroup of $Q.$ Then, \begin{align}
    |\mathcal O_k|=[Q:P_k\cap Q]
\end{align}
is divisible by $P.$ So, $p\mid|\mathcal S|$ so $p\mid|r$ but $r\equiv 1\mod p$ so this is a contradiction. 


Suppose $|G|=pq,$ and $p<q$ prime. We know $n_q(G)=1,$ i.e. $\mathrm{Syl}(G)=\{Q\},$ then $Q\triangleleft G.$ One possibilities is that $G$ is cyclic. Often, $n_p(G)=0$ unless $p\mid(q-1),$ then it is more complicated.

The significance os $q-1=$ the number of units $\mod q,$ which turns out to be the number of automorphisms of $C_q.$ (multiply each element of $C_q$ by a unit $u=(\Z/q\Z)^x.$

So this is a homomorphism $C_p\to\mathrm{Aut}(C_q).$ We can use this homomorphism to make a group that is not abelian.

\begin{definition}[Finitely Generated]
    An abelian group $G$ is \bluebf{finitely generated} if there exists a finite set $S$ such that $G=\langle S\rangle.$
\end{definition}
Examples of finitely generated abelian groups are finite abelian groups, $\Z,\Z^r$ but not $\R,S^1,\mathbb Q$.
\begin{theorem}[Fundamental Theorem of Abelian Groups]
    If $G$ is a finitely generated abelian group, then $G$ is isomorphic to a product \begin{align}
        G\cong\Z^r\times\prod_{i=1}^sC_{r_i}
    \end{align}
    where $r\in\N_0,n_i\in\N^{>1},$ and $n_{i+1}\mid n_i\forall i=1,\dots,s-1.$ Note the following:\begin{itemize}
        \item $r=0\iff G$ is finite.
        \item $G$ is cyclic $\iff r=0\land s=1$
    \end{itemize}
    Moreover, this decomposition is unique up to isomorphism.
    \begin{proof}
        The proof will come easily from another theorem later.
    \end{proof}
\end{theorem}
\begin{definition}
    In this decomposition, $r$ is called the \bluebf{free rank} of $G$ or the \bluebf{Betti number}Other Information of $G.$ The $n_i$'s are called the \bluebf{invariant factors} of $G.$
\end{definition}

Another version. Any finitely generated abelian group $G$ can be written as \begin{align}
    G=\Z^r\times\prod_{i=1}^kP_{p_i}
\end{align}
where $|G/\Z^r|=\prod_i p_i^{\alpha_i}.$ Moreover, for each $i,$ \begin{align}
    P_{p_i}=C_{p_i^{\beta_1^i}}\times C_{p_i^{\beta_2^i}}\times\cdots\times C_{p_i^{\beta_{\alpha_i}^i}}
\end{align}
where $\beta_1^i\geq \beta_2^i\geq\cdots\geq\beta_{\alpha_i}^i$ and $\beta_1^i+\beta_2^i+\cdots+\beta_{\alpha_i}^i=\alpha_i.$
The notation is awful, but idea is we can decompose $G$ into its Sylow $p$-subgroups and then decompose each Sylow $p$-subgroup into its cyclic factors. This decomposition is unique up to isomorphism.

\begin{definition}[Elementary Divisors]
    The subgroups $C_{p_i^{\beta_1^i}},\dots,C_{p_i^{\beta_{\alpha_i}^i}}$ or sometimes their orders are called the \bluebf{elementary divisors} of $G.$
\end{definition}

Semidirect product of $\R^2\rtimes SO(2).$ Translate first then rotate. e.g. $g=\left(\begin{pmatrix}
    0\\1
\end{pmatrix},\begin{pmatrix}
    0 & 1\\-1&0
\end{pmatrix}\right)$
and $g'=\left(\begin{pmatrix}
    -1\\0
\end{pmatrix},\frac{1}{\sqrt2}\begin{pmatrix}
    1&1\\-1&1
\end{pmatrix}\right).$ These are the motions of the plane that preserves lengths and angles.

\begin{theorem}
    Let $G$ be a finite group and $G_0=G.$ then construct $G_1=[G_0,G_0],\dots G_i=[G_{i-1},G_{i-1}].$  This series will always terminate, and $G$ is solvable iff $\exists r$ s.t. $G_r=\{e\}.$ \begin{proof}
        Messy, but not hard.
    \end{proof}
\end{theorem}
\begin{definition}[Upper Central Series]
    $Z_0=\{e\},Z_1(G)$ and $Z_1(G)=Z(G)$ then let $Z_2$ be a subgroup of $G$ s.t. $Z_2(G)/Z_1(G)=Z(G/Z_1(G)),$ $Z_i(G)/Z_{i-1}(G)=Z(G/Z_{i-1}(G)).$ This series will always terminate with \begin{align}
        \{e\}=Z_0\triangleleft Z_1\triangleleft Z_2\triangleleft\cdots\triangleleft Z_r=G
    \end{align}
\end{definition}
\begin{definition}[Nilpotent]
    $G$ is \bluebf{nilpotent} if $G$ is solvable and $Z_r(G)=G.$
\end{definition}
\begin{definition}[Lower Central Series]
    $G^{(0)}=G,G^{(1)}=[G,G],G^{(2)}=[G^{(0)},G^{(1)}],G^{(i)}=[G^{(0)},G^{(i-1)}].$   
\end{definition}
\begin{theorem}
    $G$ is solvable iff $\exists r$ s.t. $G^{(r)}=\{e\}.$
\end{theorem}

We have developed the following understanding of groups in order of complexity: \begin{enumerate}
    \item Trivial group $\{e\}$
    \item Cyclic group of prime order $C_p$
    \item Cyclic group $C_n$
    \item Abelian group
    \item $p$-group
    \item Nilpotent group
    \item Solvable group
\end{enumerate}

\begin{definition}[Characteristic]
    A proper subgroup $H<G$ is a \bluebf{characteristic subgroup} if $\varphi(H)=H$ for all $\varphi\in\mathrm{Alt}(G).$ (Note normal subgroups are only required to satisfy this property for inner automorphisms).
\end{definition}
\begin{proposition}
    If $H$ is normal in a characteristic subgroup of $G,$ then $H\trianglelefteq G.$ This is not true without the characteristic property.
\end{proposition}

\subsection{Nilpotent Groups}
Easy: $p$-groups are nilpotent.

(almost) easy: a product of nilpotent groups is nilpotent. (the pieces in the definition of nilpotence work ``component-wise'' in a product).

In particular, product of $p$-groups are nilpotent.

If $P$ is a $p$-group and $Q$ is a $q$-group, in $G=P\times Q,$ $P=P\times\{e\}\in\mathrm{Syl}_p(G)\triangleleft G$ and $Q=\{e\}\times Q\in\mathrm{Syl}_q(G)\triangleleft G.$ Analogously, $G=P_1\times P_2\times\cdots\times P_k$ is nilpotent iff $P_i$ is a $p_i$-group, then the $P_i$s are the Sylow $p_i$-subgroups of $G,$ and each is normal (so it is the only $p_i$ subgroup).

\begin{theorem}
    Suppose $G$ is finite, then the following are equivalent: \begin{enumerate}
        \item $G$ is nilpotent.
        \item If $H<G$ is a proper subgroup, then $H<N_G(H)$ is also a proper subgroup.
        \item If $p\mid |G|$ and $P\in \mathrm{Syl}_p(G),$ then $P$ is normal. Hence, all Sylow $p$-subgroups are normal.
        \item $G\cong P_1\times\cdots\times P_k$ where $P_i\in\mathrm{Syl}_{p_i}(G)$ and $p_1,\cdots,p_k$ are distinct primes. 
    \end{enumerate}
    \begin{proof}[Proof (hint $1\to 2$)]
        If $G$ is abelian, then the proof is trivial. So, we can assume $G$ is not abelian. Otherwise, the proof of the theorem is trivial.
    \end{proof}
\end{theorem}
Consider a finite field $\mathbb F$ and matrices over $\mathbb F$ with the form $\begin{pmatrix}
    1&a&b\\0&1&c\\0&0&1
\end{pmatrix}.$ Suppose two matrices of this form, \begin{align}
    \underbrace{\begin{pmatrix}
    1&a&b\\0&1&c\\0&0&1
\end{pmatrix}}_g\underbrace{\begin{pmatrix}
    1&x&y\\0&1&z\\0&0&1
\end{pmatrix}}_h&=\begin{pmatrix}
    1&a+x&b+az+y\\0&1&c+z\\0&0&1
\end{pmatrix}\\
hg&=\begin{pmatrix}
    1&a+x&b+cx+y\\0&1&c+z\\0&0&1
\end{pmatrix}\\
[g,h]=g\inv h\inv gh&=\begin{pmatrix}
    1&0&x\\0&1&0\\0&0&1
\end{pmatrix}
\end{align}
If $G=\left\{\begin{pmatrix}
    1&*&*\\0&1&*\\0&0&1
\end{pmatrix}\right\},$ then $[G,G]=\left\{\begin{pmatrix}
    1&0&*\\0&1&0\\0&0&1
\end{pmatrix}\right\},$ and $[G,[G,G]]=\{e\}.$ This shows $G$ is solvable. If we extend $n$ to any number beyond $3,$ the same argument holds.

\begin{theorem}[Very hard ones]
    \begin{enumerate}
        \item (Burnside) If $|G|=p^aq^b$ and $p,q$ are distinct primes, then $G$ is solvable.
        \item (Philip Hall) Suppose that for each prime $p$ dividing $|G|,$ with $|G|=p^am$ so that $gcd(m,p)=1.$ Suppose $\exists H<G$ s.t. $|H|=m$ s.t. $G=PH$ where $P\in\mathrm{Syl}_p(G).$ If this holds for all $p,$ then $G$ is solvable.
        \item (Feit--Thompson) If $|G|$ is odd, then $G$ is solvable.
        \item (Thompson) Suppose that for any $x,y\in G,$ the subgroup they generate $\langle x,y\rangle$ is solvable. Then $G$ is solvable.
    \end{enumerate}
    \begin{proof}
        The proofs of these theorems are hundreds of pages long.
    \end{proof}
\end{theorem}

\section{Rings}
Rings are a difficult topic because there are very few results that holds for all rings. To learn about rings, we need to show many things for specific cases. There are some important differences \begin{enumerate}
    \item Rings need not have multiplicative inverses.
    \item Rings need not be commutative.
    \item Rings need not have multiplicative identities.
\end{enumerate}
\begin{definition}[Ring]
    A ring $R$ is an abelian group under addition, with additive identity $0.$ A ring also have multiplication, and multiplication is associative. Also, the left and right distributive law holds: for $x,y,z\in R,$ $x(y+z)=xy+xz$ and $(x+y)z=xz+yz.$
    The ring may not have a multiplicative identity. 
\end{definition}
Examples:\begin{itemize}
    \item Any field is a ring.
    \item $\Z$
    \item The set of all matrices over a field.
    \item Polynomials of $n$ variables over a field.
    \item $\Z/n\Z$ is a finite ring. It is not a field if $n$ is not prime.
    \item If $X$ is a set, then the set of functions on $X$ with values in a field $\mathbb F$ is a commutative ring. with \begin{align}
        (f+g)(x)&=f(x)+g(x)\\
        (f\cdot g)(x)&=f(x)g(x)
    \end{align}
    \item If $X$ is a topological space, then the set of continuous functions on $X$ is a ring.
    \item $C_c(X)$ (continuous functions with compact support) is a ring.
    \item  
\end{itemize}

\end{document}

