\documentclass[a4paper]{article}
\usepackage{stdtemplate}

\title{MAT347 Abstract Algebra}
\author{Jonah Chen}
\date{}
\begin{document}
\maketitle
\sffamily
\section{Groups}
Groups are generally associated with symmetries. Consider the equilateral triangle:
\begin{center}
    \begin{tikzpicture}
        % draw a triangle and label the vertices A, B, C
        \draw (0,0) -- (4,0) -- (2,3.464) -- cycle;
        \node[above] at (2,3.464) {$A$};
        \node[below] at (4,0) {$B$};
        \node[below] at (0,0) {$C$};
    \end{tikzpicture}
\end{center}
We know that there are six symmetries of the triangle:
\begin{itemize}
    \item Identity transformation (do nothing) denoted as $\id$ or $e$
    \item Two rotations ($A\to B\to C\to A$ and $A\to C\to B\to A$)
    \item Three reflections % A <-> B
    $A \leftrightarrow B$, $A \leftrightarrow C$, $B \leftrightarrow C$
\end{itemize}
Note that these symmetries preserve the structure of the triangle, hence the composition of two symmetries must also be a symmetry. Let
\begin{itemize}
    \item $\rho$ be the rotation $A\to B\to C\to A$
    \item $\sigma$ be the reflections $B\leftrightarrow C$
\end{itemize}
Note that $\rho\sigma$ is the $A\leftrightarrow C$ reflection and $\sigma\rho$ is the $A\leftrightarrow B$ reflection. Hence they may not be commutative.

% Note that a square has eight symmetries. Then, that tells us that the square and triangle are two different shapes (this applies in general).

We also know that all symmetries can be reversed. $\alpha$ has an inverse $\alpha\inv$ such that $\alpha\alpha\inv = \alpha\inv\alpha = e$. These inspires the following definition:

\begin{definition}
    A \textbf{group} is a set $G$ with a composition
    \begin{align}
        G\times G &\to G \\
        (g, h) &\mapsto g\cdot h
    \end{align}
    Satisfying:
    \begin{itemize}
        \item Associativity: $(g\cdot h)\cdot k = g\cdot (h\cdot k)$
        \item Identity: $\exists\, e\in G$ such that $g\cdot e = e\cdot g = g$ for all $g\in G$
        \item Inverse: $\forall\, g\in G$, $\exists\, g\inv\in G$ such that $g\cdot g\inv = g\inv\cdot g = e$
    \end{itemize}
\end{definition}

Examples:
\begin{itemize}
    \item $\Z$ with $+$ is a group. It is associative, $e = 0$ and $g\inv = -g$.
    \item $\Z/n\Z$ with addition modulo $n$.
    \item If $F$ is a field, it implicitly has two group structures:
    \begin{itemize}
        \item Additive group: $(F,+)$ is a group. It is associative, $e = 0$ and $g\inv = -g$.
        \item Multiplicative group: $(F\setminus\{0\},\times)$ is a group. It is associative, $e = 1$ and $g\inv = 1/g$.
    \end{itemize}
    \item $GL(n,F)$ -- ``general linear group'' contains all invertiable $n\times n$ matrices.
    \item $SL(n,F)$ -- ``special linear group'' contains all invertiable $n\times n$ matrices with determinant $1$.
    \item $SO(n,F)$ -- ``special orthogonal group'' $=\{A\in SL(n,F)|A^t=A\inv\}$.
\end{itemize}
\subsection{Cyclic Groups}
One of the simplest groups is $\Z/n\Z$ for $n\in\N$ with the operation addition modulo n. This is known as the ``cyclic group of order $n$'' or $C_n$. i.e. for $n=8$, $5+7=4\,(\mathrm{mod}\,8)$, which we denote $\bar{5}+\bar{7}=\bar{4}$.

We know the inverse $\bar k\inv=\overline{n-k}$ for nonzero $k$ or $\bar 0\inv=\bar 0.$

Another way to express the cyclic group is $\bar k\leftrightarrow e^{2\pi i k/n}$ with multiplication operation. Then, \begin{align}
    \overline{k+n}=e^{2\pi i (k+n)/n}=e^{2\pi i k/n}e^{2\pi i n/n}=e^{2\pi i k/n}=\bar k.
\end{align}

% draw a circle and 8 equally spaced points on it
\begin{center}
    \begin{tikzpicture}
        \draw (0,0) circle (2cm);
        \foreach \x in {0,1,...,7}
        {
            % draw dot
            \draw (\x*45:2cm) circle (1pt);
        }
        % label the first point
        \node[below] at (0*45:2cm) {$1\leftrightarrow\bar 0$};
        % draw xy axis
        \draw (-3cm,0) -- (3cm,0);
        \draw (0,-3cm) -- (0,3cm);
    \end{tikzpicture}
\end{center}

\begin{definition}[Order]
    The \textbf{order} of a group $G$ is its cardinality denoted $\mathrm{ord}(G)$ or $|G|$. It could be a finite or infinite ordinal. In particular, $|C_n| = n$.
\end{definition}

\subsection{Quaternion Group}
The quaternion group $\mathbb{H}=\{\pm1,\pm i,\pm j,\pm k\}$ is a group of order $8$ with the multiplication operation. It has 

\begin{definition}[Subgroup]
    A \textbf{subgroup} of a group $G$ is a subset $H\subseteq G$ such that $H$ is a group. 
\end{definition}

\begin{definition}[Coset]
    If $G$ is a group and $H\leq G$, consider sets of the form
    \begin{equation}
        Hg=\{hg|h\in H\}
    \end{equation}
    This is a \textbf{right coset} of $H$.
\end{definition}
\begin{theorem}[Partitioning with Cosets]
    Consider $Hg$ and $Hg'$ for $g.g'\in G$. There are two cases:
    \begin{itemize}
        \item They might be disjoint: $Hg\cap Hg'=\emptyset$.
        \item They might intersect. Suppose $hg=h'g'$ for some $h,h'\in H$
        \begin{align}
            h\inv hg&=h\inv h'g'\\
            g&=h\inv h'g'\in Hg'
        \end{align}
        Similarly, $g'\in Hg$. Consider an arbitrary element of $kg\in Hg$ with $k\in H$. Then, $kg=kh\inv h'g'\in Hg'$ i.e. $Hg\leq Hg'$. Similarly, $Hg'\leq Hg$. Thus, $Hg=Hg'$.
    \end{itemize}
    The right cosets of $H$ partition $G$. In particular,
    \begin{equation}
        G = \bigsqcup Hg_i
    \end{equation}
\end{theorem}

For fixed $g$, if $hg=h'g$ for $h,h'\in H$ then $hgg\inv=h'gg\inv$ so $h=h'$. So in $Hg$, every element can be matched with an element of $H$. So, $|Hg|=|H|$.

\begin{theorem}[Lagrange]
    If $|G|<\infty$ and $H\leq G$, then $|H|\Big\vert|G|$
\end{theorem}
\begin{definition}[Index]
    For $H\leq G$, the \textbf{index} of $H$ in $G$ is $[G:H]=|G|/|H|$.
\end{definition}
If $|G|=13$, the only subgroups or $G$ are $\{e\},G$.

If $G=\Z$ and $H=2\Z$ (even numbers). Then $H+0=H$ is one coset, and $H+1=$ the odd integers is another coset. So, $\Z=(2\Z)\sqcup(2\Z+1)$.

\underline{Same for Left Cosets}
Interaction of left and right cosets? 

Consider the triangle group with rotations $e,\rho,\rho^2$ and reflections $\sigma_A, \sigma_B, \sigma_C$ Consider the subgroup $H=\{e,\sigma_A\}$.
\begin{align}
    He&=\{e,\sigma_A\}\\
    H\rho&=\{\rho,\sigma_B\}\\
    H\rho^2&=\{\rho^2,\sigma_C\}\\
    eH=\{e,\sigma_A\}\\
    \rho H&=\{\rho, \sigma_C\}\\
    \rho^2 H&=\{\rho^2,\sigma_B\}
\end{align}
Note that the left and right cosets are different. They are the same if the group is commutative.

\begin{definition}[Action]
    An \textbf{action} of a group $G$ on a set $X$ is a map \begin{align}
        G&\times X\to X \\
        (g,x)&\mapsto gx
    \end{align}
    such that 
    \begin{align}
        (gh)x&=g(hx)\\
        ex&=x
    \end{align}
\end{definition}


\end{document}
