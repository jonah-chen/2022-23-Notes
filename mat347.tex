\documentclass[a4paper]{article}
\usepackage{stdtemplate}

\title{MAT347 Abstract Algebra}
\author{Jonah Chen}
\date{}
\begin{document}
\maketitle
\sffamily
\section{Groups}
Groups are generally associated with symmetries. Consider the equilateral triangle:
\begin{center}
    \begin{tikzpicture}
        % draw a triangle and label the vertices A, B, C
        \draw (0,0) -- (4,0) -- (2,3.464) -- cycle;
        \node[above] at (2,3.464) {$A$};
        \node[below] at (4,0) {$B$};
        \node[below] at (0,0) {$C$};
    \end{tikzpicture}
\end{center}
We know that there are six symmetries of the triangle:
\begin{itemize}
    \item Identity transformation (do nothing) denoted as $\id$ or $e$
    \item Two rotations ($A\to B\to C\to A$ and $A\to C\to B\to A$)
    \item Three reflections % A <-> B
    $A \leftrightarrow B$, $A \leftrightarrow C$, $B \leftrightarrow C$
\end{itemize}
Note that these symmetries preserve the structure of the triangle, hence the composition of two symmetries must also be a symmetry. Let
\begin{itemize}
    \item $\rho$ be the rotation $A\to B\to C\to A$
    \item $\sigma$ be the reflections $B\leftrightarrow C$
\end{itemize}
Note that $\rho\sigma$ is the $A\leftrightarrow C$ reflection and $\sigma\rho$ is the $A\leftrightarrow B$ reflection. Hence they may not be commutative.

% Note that a square has eight symmetries. Then, that tells us that the square and triangle are two different shapes (this applies in general).

We also know that all symmetries can be reversed. $\alpha$ has an inverse $\alpha\inv$ such that $\alpha\alpha\inv = \alpha\inv\alpha = e$. These inspires the following definition:

\begin{definition}
    A \textbf{group} is a set $G$ with a composition
    \begin{align}
        G\times G &\to G \\
        (g, h) &\mapsto g\cdot h
    \end{align}
    Satisfying:
    \begin{itemize}
        \item Associativity: $(g\cdot h)\cdot k = g\cdot (h\cdot k)$
        \item Identity: $\exists\, e\in G$ such that $g\cdot e = e\cdot g = g$ for all $g\in G$
        \item Inverse: $\forall\, g\in G$, $\exists\, g\inv\in G$ such that $g\cdot g\inv = g\inv\cdot g = e$
    \end{itemize}
\end{definition}

Examples:
\begin{itemize}
    \item $\Z$ with $+$ is a group. It is associative, $e = 0$ and $g\inv = -g$.
    \item $\Z/n\Z$ with addition modulo $n$.
    \item $SL(n, F)$ all $n\times n$ matrices with determinant $1$ over a field $F$. 
\end{itemize}




\end{document}